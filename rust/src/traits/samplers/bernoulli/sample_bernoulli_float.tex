\documentclass{article}
% common styling and macros shared by all proof files

\usepackage[top=1in, right=1in, left=1in, bottom=1.5in]{geometry}

\usepackage{amsmath,amsthm,amsfonts,amssymb,amscd}
\usepackage{listings}
\usepackage{hyperref}
\usepackage{xcolor}
\usepackage{xr}

\usepackage{enumerate} 
\usepackage{physics}
\usepackage{fancyhdr}
\usepackage{hyperref}
\usepackage{graphicx}
\usepackage{tcolorbox}
\usepackage{catchfile}
\usepackage{pdftexcmds}
\usepackage[T1]{fontenc}

% hyperref
\hypersetup{
  colorlinks=true,
  linkcolor=blue,
  linkbordercolor={0 0 1}
}

% \contrib macro to indicate inclusion in "contrib".
\usepackage{tcolorbox}
\newtcolorbox{contrib_box}{colback=red!5!white,colframe=red!75!black}
\newcommand{\contrib}{{\begin{contrib_box}This proof resides in \textbf{``contrib''} because it has not completed the vetting process.\end{contrib_box}}} 

% asOfCommit macro to version a code dependency. Arguments:
%    #1: relative path to file you are dependent on
%    #2: commit hash it was last edited. If outdated, this should be the second hash in the footnoote. Otherwise,
%            git log -n 1 --pretty=format:%h -- path/to/file.rs
\makeatletter
\ifnum\pdf@shellescape=1
   % "private" command that builds a link to a blob
  \newcommand{\linkOpendpBlob}[3]{%
    \href{https://github.com/opendp/opendp/blob/#1/#2#3}{\path{#3} at commit #1}}

  % latex macro expansion has a separate phase for \input evaluation
  %     immediately evaluate a command to write a temp file to ./out containing the current directory
  \immediate\write18{[ ! -f out/cwd.txt ] && (mkdir -p out && git rev-parse --show-prefix | sed "s|_|\@backslashchar\@backslashchar\@backslashchar_|g" > out/cwd.txt)}
  %     ...and then retrieve the current working directory by loading the temp file
  \CatchFileDef\GitWorkingDir{out/cwd.txt}{\endlinechar=-1}

  % command for building the (up to date) or (outdated) status
  \newcommand{\fileStatus}[2]{%
  \setbox0=\hbox{\input|"git --no-pager log -n1 --pretty='\@percentchar H' #1 | grep -E '^#2.*'"\unskip}\ifdim\wd0=0pt
        (outdated\footnote{See new changes with \texttt{git diff #2..\input|"git --no-pager log -n1 --pretty='\@percentchar h' #1" \GitWorkingDir\path{#1}}})\else
        (up to date)\fi
  }

  \newcommand{\asOfCommit}[2]{%
      % permalink the target
      \linkOpendpBlob{#2}{\GitWorkingDir}{#1}
      % conditionally add (outdated) or (up to date) depending on matching commit hash
      \fileStatus{#1}{#2}%
  }
\else
  % simplified command if shell-escape not enabled
  \newcommand{\asOfCommit}[2]{#1 at commit #2 (unknown status\footnote{Shell-escape is not enabled. Enable \texttt{--shell-escape} to check if this proof is up-to-date with the code.})}
\fi
\makeatother

% \vettingPR macro to link a PR. Arguments:
%    #1: PR number
\newcommand{\vettingPR}[1]{\href{https://github.com/opendp/opendp/pull/#1}{Pull Request \##1}}

% \OpenDPVersion macro to get library version. Mainly used for \rustdoc
\makeatletter
\ifnum\pdf@shellescape=1
  % latex macro expansion has a separate phase for \input evaluation
  %     immediately evaluate a command to write a temp file to ./out containing the current version
  \immediate\write18{[ ! -f out/version.txt ] && (mkdir -p out && head -n 1 `git rev-parse --show-toplevel`/VERSION | sed "s|0.0.0+development|latest|g" > out/version.txt)}
  %     ...and then retrieve the version by loading the temp file
  \CatchFileDef\OpenDPVersion{out/version.txt}{\endlinechar=-1}
\else
  % if shell commands are not enabled, just claim latest
  \newcommand{\OpenDPVersion}{latest}
\fi
\makeatother

% for links to rustdoc items in OpenDP. Arguments:
%    #1: path to item, and designation as trait, struct, fn, etc.
%    #2: item name
\newcommand{\rustdoc}[2]{\href{https://docs.rs/opendp/\OpenDPVersion/opendp/#1.#2.html}{\texttt{#2}}}

% for links to external dependencies. Arguments:
%    #1: crate name
%    #2: path to item, and designation as trait, struct, fn, etc.
%    #3: item name
\newcommand{\docsrs}[3]{\href{https://docs.rs/#1/latest/#1/#2.#3.html}{\texttt{#3}}}

% minted (pseudocode)
\definecolor{codegreen}{rgb}{0,0.6,0}
\definecolor{codegray}{rgb}{0.5,0.5,0.5}
\definecolor{codepurple}{rgb}{0.58,0,0.82}
\definecolor{backcolour}{rgb}{0.95,0.95,0.92}

\lstdefinestyle{mystyle}{
    backgroundcolor=\color{backcolour},   
    commentstyle=\color{codegreen},
    keywordstyle=\color{magenta},
    numberstyle=\tiny\color{codegray},
    stringstyle=\color{codepurple},
    basicstyle=\ttfamily\footnotesize,
    breakatwhitespace=false,         
    breaklines=true,                 
    captionpos=b,                    
    keepspaces=true,                 
    numbers=left,                    
    numbersep=5pt,                  
    showspaces=false,                
    showstringspaces=false,
    showtabs=false,                  
    tabsize=2
}

\lstset{style=mystyle}

% common commands
\theoremstyle{definition}
\newtheorem{theorem}{Theorem}[section]
\newtheorem{lemma}[theorem]{Lemma}
\newtheorem{definition}[theorem]{Definition}
\newtheorem{warning}{Warning}
\newtheorem{corollary}{Corollary}
\newtheorem{proposition}{Proposition}
\newtheorem{remark}{Remark}
\newtheorem{observation}{Observation}
\newtheorem{note}{Note}

\newcommand{\vicki}[1]{{ {\color{olive}{(vicki)~#1}}}}
\newcommand{\hanwen}[1]{{ {\color{purple}{(hanwen)~#1}}}}
\newcommand{\zach}[1]{{ {\color{red}{(zach)~#1}}}}

\newcommand{\MultiSet}{\mathrm{MultiSet}}
\newcommand{\len}{\mathrm{len}}
\newcommand{\din}{\texttt{d\_in}}
\newcommand{\dout}{\texttt{d\_out}}
\newcommand{\T}{\texttt{T} }
\newcommand{\F}{\texttt{F} }
\newcommand{\Map}{\texttt{Map}}
\newcommand{\X}{\mathcal{X}}
\newcommand{\Y}{\mathcal{Y}}
\newcommand{\True}{\texttt{True}}
\newcommand{\False}{\texttt{False}}
\newcommand{\clamp}{\texttt{clamp}}
\newcommand{\function}{\texttt{function}}
\newcommand{\float}{\texttt{float }}
\newcommand{\questionc}[1]{\textcolor{red}{\textbf{Question:} #1}}


\newcommand{\transformationTheorem}[2]{%
  \begin{theorem}
    For every setting of the input parameters #1 to #2 such that the given preconditions
    hold, #2 raises an exception (at compile time or run time) or returns a valid transformation. A valid transformation has the following properties:
    \begin{enumerate}
        \item \textup{(Appropriate output domain).} 
        For every element $v$ in \texttt{input\_domain}, $\function(v)$ is in \texttt{output\_domain} or raises a data-independent runtime exception.
        
        \item \textup{(Domain-metric compatibility).} 
        The domain \texttt{input\_domain} matches one of the possible domains listed in the definition of \texttt{input\_metric}, 
        and likewise \texttt{output\_domain} matches one of the possible domains listed in the definition of \texttt{output\_metric}.
        
        \item \textup{(Stability guarantee).} 
        For every pair of elements $u, v$ in \texttt{input\_domain} and for every pair $(\din, \dout)$, 
        where \din\ has the associated type for \texttt{input\_metric} and \dout\ has the associated type for \\ \texttt{output\_metric}, 
        if $u, v$ are \din-close under \texttt{input\_metric} and $\texttt{stability\_map}(\din) \leq \dout$, 
        then $\function(u), \function(v)$ are $\dout$-close under \texttt{output\_metric}.
    \end{enumerate}
  \end{theorem}
}


\title{\texttt{fn sample\_bernoulli\_float}}
\author{Vicki Xu, Hanwen Zhang, Zachary Ratliff}

\begin{document}
\maketitle

\contrib
\begin{tcolorbox}
    \begin{warning}[Code is not constant-time]
     \texttt{sample\_bernoulli\_float} takes in a boolean \texttt{constant\_time} parameter to protect against timing attacks on the Bernoulli sampling procedure. However, the current implementation does not guard against other types of timing side-channels that can break differential privacy, e.g., non-constant time code execution due to branching.
    \end{warning}
\end{tcolorbox}

\subsection*{PR History}
\begin{itemize}
    \item \vettingPR{473}
\end{itemize}

This document proves that the implementation of \rustdoc{traits/samplers/bernoulli/fn}{sample\_bernoulli\_float} in \asOfCommit{mod.rs}{f5bb719} 
satisfies its proof definition.

\texttt{sample\_bernoulli\_float} considers the binary expansion of \texttt{prob} into an infinite sequence $\texttt{a\_i}$, 
like so: $\texttt{prob} = \sum_{i = 0}^{\infty} \frac{a_i}{2^{i + 1}}$. 
The algorithm samples $I \sim Geom(0.5)$ using an internal function \rustdoc{traits/samplers/geometric/fn}{sample\_geometric\_buffer}, then returns $a_I$. 

\subsection{Hoare Triple}
\subsubsection*{Preconditions}
\begin{itemize}
    \item \textbf{User-specified types:}
    \begin{itemize}
        \item Variable \texttt{prob} must be of type \texttt{T}
        \item Variable \texttt{constant\_time} must be of type \texttt{bool}
        \item Type \texttt{T} has trait \rustdoc{traits/trait}{Float}. 
            \texttt{Float} implies there exists an associated type \texttt{T::Bits} (defined in \rustdoc{traits/trait}{FloatBits}) that captures the underlying bit representation of \texttt{T}.
        \item Type \texttt{T::Bits} has traits \texttt{PartialOrd} and \texttt{ExactIntCast<usize>}
        \item Type \texttt{usize} has trait \texttt{ExactIntCast<T::Bits>}
    \end{itemize}
\end{itemize}

\subsubsection*{Pseudocode}

\lstinputlisting[language=Python,firstline=2,escapechar=|]{./pseudocode/sample_bernoulli_float.py}

\subsubsection*{Postcondition}

\begin{definition}
    \label{sample-bernoulli}
    For any setting of the input parameters
    \texttt{prob} of type \texttt{T} restricted to $[0, 1]$,
    and \texttt{constant\_time} of type \texttt{bool},
    \texttt{sample\_bernoulli\_float} either
    \begin{itemize}
        \item raises an exception if there is a lack of system entropy,
        \item returns \texttt{out} where \texttt{out} is $\top$ with probability \texttt{prob}, otherwise $\bot$.
    \end{itemize}
     If \texttt{constant\_time} is set, the implementation's runtime is constant.    
\end{definition}

\subsection{Proof}
\begin{proof} 
To show the correctness of \texttt{sample\_bernoulli} we observe first that the base-2 representation of \texttt{prob} is of the form 
\[
\texttt{leading\_zeroes || implicit\_bit || mantissa || trailing\_zeroes}
\]
and is represented \emph{exactly} as a normal floating-point number. The \href{https://en.wikipedia.org/wiki/IEEE_754}{IEEE-754 standard} represents a normal floating-point number using an exponent $E$, and a mantissa $m$, using a base-2 analog of scientific notation. 

\begin{definition}[Floating-Point Number]
A $(k,\ell)$-bit floating-point number $z$ is represented as
\[
z = (-1)^s \cdot (B.M) \cdot (2^E) 
\]
where
\begin{itemize}
    \item $s$ is used to represent the \emph{sign} of $z$
    \item $B$ is the implicit bit; $1$ for normal floating-point numbers and $0$ for subnormal floating point numbers
    \item $M \in \{0,1\}^k$ is a $k$-bit string representing the part of the mantissa to the right of the radix point, i.e.,
    \[
    1.M = \sum_{i = 1}^k M_i2^{-i}
    \]
    \item $E \in \mathbb{Z}$ represents the \emph{exponent} of $z$. When $\ell$ bits are allocated to representing $E$, then $E \in [-(2^{\ell - 1} - 2), 2^{\ell - 1}] \cap \mathbb{Z}$. Note that the range of $E$ is $2^\ell - 2$ rather than $2^\ell$ as the remaining to numbers are used to represent special floating point values. When $E = -(2^{\ell -1} - 2)$, then the floating point number is considered \emph{subnormal}. 
\end{itemize} 
\end{definition}

We now use the technique for \href{https://web.archive.org/web/20160418185834/https://amakelov.wordpress.com/2013/10/10/arbitrarily-biasing-a-coin-in-2-expected-tosses/}{arbitrarily biasing a coin in 2 expected tosses} as a building block. Recall that we can represent the probability $\texttt{prob}$ as $\texttt{prob} = \sum_{i = 0}^\infty \frac{a_i}{2^{i + 1}}$ for $a_i \in \{0, 1\}$, where $a_i$ is the zero-indexed $i$-th significant bit in the binary expansion of $\texttt{prob}$. Then let $I \sim Geom(0.5)$ and observe that the random variable $a_I$ is an exact Bernoulli sample with probability $\texttt{prob}$ since $P(a_I = 1) = \sum_{i = 0}^\infty P(a_i = 1|I = i)P(I = i) = \sum_{i = 1}^\infty a_i \cdot \frac{1}{2^{i + 1}} = \texttt{prob}$. It is therefore sufficient to show that for any $(k,\ell)$-bit float $\texttt{prob} = \sum_{i = 0}^\infty \frac{a_i}{2^{i + 1}}$, \texttt{sample\_bernoulli} returns the value $a_I$ with $I \sim Geom(0.5)$.

First, we observe that by line \ref{line:1check}, if $\texttt{prob} = 1.0$ then \texttt{sample\_bernoulli} returns \texttt{true} which is correct by definition of a Bernoulli random variable. Otherwise, the variable \texttt{max\_coin\_flips} is computed to be the value $\texttt{T::EXPONENT\_BIAS} + \texttt{T::MANTISSA\_BITS}$ which equals $2^{\ell - 1} - 1 + k$ for any $(k,\ell)$-bit float. Since \texttt{prob} has finite precision, there is some $j$ for which $a_i = 0$ for all $i > j$. For all $(k,\ell)$-bit floating-point numbers, $j \le 2^{\ell - 1} - 1 + k$ by definition. Then \texttt{sample\_bernoulli} calls \texttt{sample\_geometric\_buffer} with a buffer of length $\lceil \frac{\texttt{max\_coin\_flips}}{8}\rceil$ bytes (as shown in lines \ref{line:maxcoinflips} and \ref{line:bufferlen}) which returns  $\texttt{None}$ if and only if $I > {8\cdot \lceil \frac{2^{\ell - 1} -1 + k}{8}\rceil}$, where $I \sim Geom(0.5)$ (by Theorem 2.1). In this case, since $I > j$ this index appears in the \texttt{trailing\_zeroes} part of the binary expansion of \texttt{prob} and should always return \texttt{false}, i.e., $a_I = 0$ for all $I > j$. We can therefore restrict our attention to when \texttt{sample\_geometric\_buffer} returns an index $I \le \texttt{max\_coin\_flips}$ and show that \texttt{sample\_bernoulli} always returns $a_I$. 

Assuming that \texttt{sample\_geometric\_buffer} returns some $I < j$,  \texttt{sample\_bernoulli} computes the number of leading zeroes in the binary expansion of \texttt{prob} to be $\texttt{leading\_zeroes} = \texttt{T::EXPONENT\_BIAS} - 1 - \texttt{raw\_exponent(prob)}$, where \texttt{raw\_exponent(prob)} is the value stored in the $\ell$ bits of the exponent. This value is correct by the specification of a $(k,\ell)$-bit float.  \texttt{sample\_bernoulli} then matches on the value \texttt{first\_heads\_index} corresponding to $I \sim Geom(0.5)$ returned by the function \texttt{sample\_geometric\_buffer}: \\

\noindent\textbf{Case 1} ($\texttt{first\_heads\_index} < \texttt{leading\_zeroes}$). \\
\noindent This corresponds to \texttt{sample\_geometric\_buffer} returning a value $I$ such that $a_I$ indexes into the \texttt{leading\_zeroes} part of the  \texttt{prob} variable's binary expansion. Therefore, for any $I < \texttt{leading\_zeroes}$, it follows that $a_I = 0$ and we should return \texttt{false}. In this case, \texttt{sample\_bernoulli} returns \texttt{false}.\\

\noindent\textbf{Case 2} ($\texttt{first\_heads\_index} == \texttt{leading\_zeroes}$). \\
\noindent This corresponds to \texttt{sample\_geometric\_buffer} returning a value $I$ such that $a_I$ indexes into the \texttt{implicit\_bit} part of the  \texttt{prob} variable's binary expansion. When \texttt{prob} is a normal floating point value, i.e., $E \ne -(2^{\ell -1} - 2)$ then the implicit bit $a_I = 1$. Otherwise, when \texttt{prob} is a subnormal floating point value, i.e., $E = -(2^{\ell - 1} - 2)$, the implicit bit $a_I = 0$. Since \texttt{raw\_exponent(prob)} corresponds to the exponent $E$ for any $(k,\ell)$-bit floating point number \texttt{prob}, \texttt{sample\_bernoulli} returns \texttt{true} when $\texttt{raw\_exponent(prob)} \ne 0$ and \texttt{false} otherwise. \\


\noindent\textbf{Case 3} ($\texttt{leading\_zeroes}  + \texttt{T::MANTISSA\_BITS} < I$). This corresponds to the case where  \texttt{sample\_geometric\_buffer} returns a value $I$ where $I > j$, but $I < \texttt{max\_coin\_flips}$ and therefore $a_I$ indexes into the trailing zeroes. In this case, \texttt{sample\_bernoulli} returns \texttt{false} since $a_I = 0$ for all bits in the \texttt{trailing\_zeroes} part of \texttt{prob}'s binary expansion. \\

\noindent\textbf{Case 4} ($ \texttt{leading\_zeroes} < \texttt{first\_heads\_index}  <  \texttt{leading\_zeroes}  + \texttt{T::MANTISSA\_BITS}$). \\
\noindent This corresponds to \texttt{sample\_geometric\_buffer} returning a value $I$ such that $a_I$ indexes into the \texttt{mantissa} part of the  \texttt{prob} variable's binary expansion. In this case, 
\texttt{sample\_bernoulli}  left-shifts the value \texttt{1} by  $(\texttt{MANTISSA\_BITS + leading\_zeroes - first\_heads\_index})$ digits, the index into the mantissa corresponding to the digit $a_I$ in the binary representation of \texttt{prob}. Since the operation between the left-shifted \texttt{1} and the binary representation of \texttt{prob} at that position is a bitwise AND, if the bit in question is 1 (matching the left-shifted \texttt{1}), \texttt{sample\_bernoulli} will return \texttt{true}. Otherwise, \texttt{sample\_bernoulli} will return \texttt{false}. \\


\noindent Therefore, for any value of \texttt{prob}, the function \texttt{sample\_bernoulli} either raises an exception or returns the value \texttt{true} with probability exactly $\texttt{prob}$.  
\end{proof}

\end{document}
