\documentclass{article}
\documentclass[11pt,letterpaper]{article}
\usepackage[top=1in, right=1in, left=1in, bottom=1.5in]{geometry}

\usepackage{amsmath,amsthm,amsfonts,amssymb,amscd}
\usepackage{listings}
\usepackage{hyperref}
\usepackage{xcolor}
\usepackage{xr}
\usepackage[outputdir=./out]{minted}

\hypersetup{
  colorlinks=true,
  linkcolor=blue,
  linkbordercolor={0 0 1}
}


% Pseudocode
\definecolor{codegreen}{rgb}{0,0.6,0}
\definecolor{codegray}{rgb}{0.5,0.5,0.5}
\definecolor{codepurple}{rgb}{0.58,0,0.82}
\definecolor{backcolour}{rgb}{0.95,0.95,0.92}

\setminted[python]{
    firstline=2,
    firstnumber=1,
    linenos=true,
    numbersep=5pt,
    gobble=0,
    frame=leftline,
    framerule=0.4pt,
    framesep=2mm,
    funcnamehighlighting=true,
    tabsize=4,
    obeytabs=false,
    mathescape=false
    showspaces=false,
    showtabs =false,
    texcomments=true,
    fontsize=\small
}
% Commands
\newcommand{\dependsOn}[2]{\immediate\write18{
    mkdir -p out_deps && 
    git show #1:#2 > ./out_deps/$(basename #2)
}}

\newcommand{\MultiSet}{\mathrm{MultiSet}}
\newcommand{\MultiSets}{\mathrm{MultiSets}}
\newcommand{\len}{\mathrm{len}}
\newcommand{\din}{\mathrm{d_{in}}}
\newcommand{\dout}{\mathrm{d_{out}}}
\newcommand{\Relation}{\mathrm{Relation}}
\newcommand{\question}[1]{\textcolor{red}{\textbf{Question:} #1}}
\newcommand{\ellOne}{\mathrm{\ell_1}}
\newcommand{\maxUsize}{\texttt{usize::MAX}}
\newcommand{\function}{\texttt{function}}
\newcommand{\True}{\texttt{True}}
\newcommand{\todo}[1]{{\begin{center} \textcolor{teal}{{\huge TODO:} #1}\end{center}}}
\newcommand{\notebig}[1]{{ \textcolor{red}{{\huge Note:} #1}}}

\newtheorem{theorem}{Theorem}[section]
\newtheorem{lemma}[theorem]{Lemma}

\theoremstyle{definition}
\newtheorem{remark}{Remark}
\newtheorem{definition}[theorem]{Definition}
\newtheorem{observation}{Observation}
\newtheorem{note}{Note}
\newtheorem{hope}{Hope}
\newtheorem{warning}{Warning}
\newtheorem{problem}{Problem}
\newtheorem{fear}{Fear}


\title{\texttt{fn sample\_discrete\_laplace}}
\author{Michael Shoemate}

\begin{document}
\maketitle

\contrib
Proves soundness of \texttt{fn sample\_discrete\_laplace} in \asOfCommit{mod.rs}{0be3ab3e6}.
This proof is an adaptation of \href{https://arxiv.org/pdf/2004.00010.pdf#subsection.5.2}{subsection 5.2} of \cite{CKS20}.

\section{Vetting history}
\begin{itemize}
    \item \vettingPR{519}
\end{itemize}

\section{Pseudocode}
\subsubsection*{Precondition}
$\texttt{scale} \in \mathbb{Q} \land \texttt{scale} > 0$

\subsubsection*{Implementation}        
\begin{lstlisting}[language=Python]
def sample_discrete_laplace(scale) -> int:
    if (scale == 0):
        return 0
        
    inv_scale = recip(scale)
    
    while True:
        sign = sample_standard_bernoulli()
        magnitude = sample_geometric_exp_fast(inv_scale)
        
        if sign or magnitude != 0:
            if sign:
                return -magnitude
            else:
                return magnitude
\end{lstlisting}

\subsubsection*{Postcondition}
\texttt{sample\_discrete\_laplace} returns an integer sample from $\mathcal{L}_\mathbb{Z}(0, scale)$, or an error propagated from \texttt{sample\_geometric\_exp\_fast} or \texttt{sample\_standard\_bernoulli}.

\section{Proof}

First, define the target distribution.

\begin{definition}
    (Discrete Laplace). Let $\mu, \sigma \in \mathbb{R}$ with $\sigma > 0$. The discrete laplace distribution with location $\mu$ and scale $s$ is denoted $\mathcal{L}_\mathbb{Z}(\mu, s)$. It is a probability distribution supported on the integers and defined by \cite{BV17}
\begin{equation*}
    \forall x \in \mathbb{Z} \quad  P[X = x] = \frac{e^{-1/s} - 1}{e^{-1/s} + 1} e^{-|x|/s} \quad \text{where } X \sim \mathcal{L}_\mathbb{Z}(\mu, s)
\end{equation*}
\end{definition}

% We must show that the return value of \texttt{sample\_discrete\_laplace}, denoted $y$, conditioned on not returning an error, is a sample from $\mathcal{L}_\mathbb{Z}(0, scale)$.

\begin{lemma}
\label{P_B_Y_ne_T_0}
Let $B \sim Bernoulli(1/2)$ and $Y \sim Geometric(1 - e^{-1/s})$ for some $s > 0$. Then $P[(B, Y) \neq (\top, 0)] = \frac{1}{2} (e^{-1/s} + 1)$\cite{CKS20}.
\end{lemma}

\begin{proof}
\begin{align*}
    P[(B, Y) \neq (\top, 0)] &= P[B = \top, Y > 0] + P[B = \bot] && \text{by LOTP} \\
    &= P[B = \top] P[Y > 0] + P[B = \bot] && \text{by independence of B, Y} \\
    &= \frac{1}{2} e^{-1/s} + \frac{1}{2} \\
    &= \frac{1}{2} (e^{-1/s} + 1)
\end{align*}
\end{proof}

\begin{theorem}
\label{P_Lx_BY_ne_T0}
Given random variables $B \sim Bernoulli(1/2)$ and $Y \sim Geometric(1 - e^{-1/s})$, define $X|_{B=\top} = Y$, and $X|_{B=\bot} = -Y$. If $(B, Y) \neq (\top, 0)$, then $X \sim \mathcal{L}_\mathbb{Z}(0, scale)$. That is, $P[X = x | (B, Y) \neq (\top, 0)] = \frac{1 - e^{-1/\sigma}}{1 + e^{-1/\sigma}} e^{-|x|/\sigma}$ for any $x \in \mathbb{Z}$\cite{CKS20}.
\end{theorem}

\begin{proof}
\begin{align*}
P[X = x | (B, Y) \neq (\top, 0)] &= \frac{P[X = x, (B, Y) \neq (\top, 0)]}{P[(B, Y) \neq (\top, 0)]} \\
    &= \frac{P[X = |x|, B = \mathbb{I}[x < 0]]}{P[(B, Y) \neq (\top, 0)]} && \text{since x = }\pm y \\
    &= \frac{P[X = |x|] P[B = \mathbb{I}[x < 0]]}{P[(B, Y) \neq (\top, 0)]} && \text{by independence of B, Y} \\
    &= \frac{P[X = |x|] \frac{1}{2}}{\frac{1}{2} (e^{-1/s} + 1)} && \text{by } \ref{P_B_Y_ne_T_0} \\
    &= \frac{1 - e^{-1/s}}{1 + e^{-1/s}} e^{-|x|/s} \\
\end{align*}
\end{proof}

Since \texttt{sign} is a realization of $B$, \texttt{magnitude} is a realization of $Y \sim Geometric(1 - e^{-1/scale})$, and the return is conditioned on $(B, Y) \neq (\top, 0)$ and not returning an error, then by \ref{P_Lx_BY_ne_T0} the function's return value is a realization of $\mathcal{L}_\mathbb{Z}(0, scale)$.


\bibliographystyle{alpha}
\bibliography{mod}

\end{document}