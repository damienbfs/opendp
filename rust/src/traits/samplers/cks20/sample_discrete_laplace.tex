\documentclass{article}
% common styling and macros shared by all proof files

\usepackage[top=1in, right=1in, left=1in, bottom=1.5in]{geometry}

\usepackage{amsmath,amsthm,amsfonts,amssymb,amscd}
\usepackage{listings}
\usepackage{hyperref}
\usepackage{xcolor}
\usepackage{xr}

\usepackage{enumerate} 
\usepackage{physics}
\usepackage{fancyhdr}
\usepackage{hyperref}
\usepackage{graphicx}
\usepackage{tcolorbox}
\usepackage{catchfile}
\usepackage{pdftexcmds}
\usepackage[T1]{fontenc}

% hyperref
\hypersetup{
  colorlinks=true,
  linkcolor=blue,
  linkbordercolor={0 0 1}
}

% \contrib macro to indicate inclusion in "contrib".
\usepackage{tcolorbox}
\newtcolorbox{contrib_box}{colback=red!5!white,colframe=red!75!black}
\newcommand{\contrib}{{\begin{contrib_box}This proof resides in \textbf{``contrib''} because it has not completed the vetting process.\end{contrib_box}}} 

% asOfCommit macro to version a code dependency. Arguments:
%    #1: relative path to file you are dependent on
%    #2: commit hash it was last edited. If outdated, this should be the second hash in the footnoote. Otherwise,
%            git log -n 1 --pretty=format:%h -- path/to/file.rs
\makeatletter
\ifnum\pdf@shellescape=1
   % "private" command that builds a link to a blob
  \newcommand{\linkOpendpBlob}[3]{%
    \href{https://github.com/opendp/opendp/blob/#1/#2#3}{\path{#3} at commit #1}}

  % latex macro expansion has a separate phase for \input evaluation
  %     immediately evaluate a command to write a temp file to ./out containing the current directory
  \immediate\write18{[ ! -f out/cwd.txt ] && (mkdir -p out && git rev-parse --show-prefix | sed "s|_|\@backslashchar\@backslashchar\@backslashchar_|g" > out/cwd.txt)}
  %     ...and then retrieve the current working directory by loading the temp file
  \CatchFileDef\GitWorkingDir{out/cwd.txt}{\endlinechar=-1}

  % command for building the (up to date) or (outdated) status
  \newcommand{\fileStatus}[2]{%
  \setbox0=\hbox{\input|"git --no-pager log -n1 --pretty='\@percentchar H' #1 | grep -E '^#2.*'"\unskip}\ifdim\wd0=0pt
        (outdated\footnote{See new changes with \texttt{git diff #2..\input|"git --no-pager log -n1 --pretty='\@percentchar h' #1" \GitWorkingDir\path{#1}}})\else
        (up to date)\fi
  }

  \newcommand{\asOfCommit}[2]{%
      % permalink the target
      \linkOpendpBlob{#2}{\GitWorkingDir}{#1}
      % conditionally add (outdated) or (up to date) depending on matching commit hash
      \fileStatus{#1}{#2}%
  }
\else
  % simplified command if shell-escape not enabled
  \newcommand{\asOfCommit}[2]{#1 at commit #2 (unknown status\footnote{Shell-escape is not enabled. Enable \texttt{--shell-escape} to check if this proof is up-to-date with the code.})}
\fi
\makeatother

% \vettingPR macro to link a PR. Arguments:
%    #1: PR number
\newcommand{\vettingPR}[1]{\href{https://github.com/opendp/opendp/pull/#1}{Pull Request \##1}}

% \OpenDPVersion macro to get library version. Mainly used for \rustdoc
\makeatletter
\ifnum\pdf@shellescape=1
  % latex macro expansion has a separate phase for \input evaluation
  %     immediately evaluate a command to write a temp file to ./out containing the current version
  \immediate\write18{[ ! -f out/version.txt ] && (mkdir -p out && head -n 1 `git rev-parse --show-toplevel`/VERSION | sed "s|0.0.0+development|latest|g" > out/version.txt)}
  %     ...and then retrieve the version by loading the temp file
  \CatchFileDef\OpenDPVersion{out/version.txt}{\endlinechar=-1}
\else
  % if shell commands are not enabled, just claim latest
  \newcommand{\OpenDPVersion}{latest}
\fi
\makeatother

% for links to rustdoc items in OpenDP. Arguments:
%    #1: path to item, and designation as trait, struct, fn, etc.
%    #2: item name
\newcommand{\rustdoc}[2]{\href{https://docs.rs/opendp/\OpenDPVersion/opendp/#1.#2.html}{\texttt{#2}}}

% for links to external dependencies. Arguments:
%    #1: crate name
%    #2: path to item, and designation as trait, struct, fn, etc.
%    #3: item name
\newcommand{\docsrs}[3]{\href{https://docs.rs/#1/latest/#1/#2.#3.html}{\texttt{#3}}}

% minted (pseudocode)
\definecolor{codegreen}{rgb}{0,0.6,0}
\definecolor{codegray}{rgb}{0.5,0.5,0.5}
\definecolor{codepurple}{rgb}{0.58,0,0.82}
\definecolor{backcolour}{rgb}{0.95,0.95,0.92}

\lstdefinestyle{mystyle}{
    backgroundcolor=\color{backcolour},   
    commentstyle=\color{codegreen},
    keywordstyle=\color{magenta},
    numberstyle=\tiny\color{codegray},
    stringstyle=\color{codepurple},
    basicstyle=\ttfamily\footnotesize,
    breakatwhitespace=false,         
    breaklines=true,                 
    captionpos=b,                    
    keepspaces=true,                 
    numbers=left,                    
    numbersep=5pt,                  
    showspaces=false,                
    showstringspaces=false,
    showtabs=false,                  
    tabsize=2
}

\lstset{style=mystyle}

% common commands
\theoremstyle{definition}
\newtheorem{theorem}{Theorem}[section]
\newtheorem{lemma}[theorem]{Lemma}
\newtheorem{definition}[theorem]{Definition}
\newtheorem{warning}{Warning}
\newtheorem{corollary}{Corollary}
\newtheorem{proposition}{Proposition}
\newtheorem{remark}{Remark}
\newtheorem{observation}{Observation}
\newtheorem{note}{Note}

\newcommand{\vicki}[1]{{ {\color{olive}{(vicki)~#1}}}}
\newcommand{\hanwen}[1]{{ {\color{purple}{(hanwen)~#1}}}}
\newcommand{\zach}[1]{{ {\color{red}{(zach)~#1}}}}

\newcommand{\MultiSet}{\mathrm{MultiSet}}
\newcommand{\len}{\mathrm{len}}
\newcommand{\din}{\texttt{d\_in}}
\newcommand{\dout}{\texttt{d\_out}}
\newcommand{\T}{\texttt{T} }
\newcommand{\F}{\texttt{F} }
\newcommand{\Map}{\texttt{Map}}
\newcommand{\X}{\mathcal{X}}
\newcommand{\Y}{\mathcal{Y}}
\newcommand{\True}{\texttt{True}}
\newcommand{\False}{\texttt{False}}
\newcommand{\clamp}{\texttt{clamp}}
\newcommand{\function}{\texttt{function}}
\newcommand{\float}{\texttt{float }}
\newcommand{\questionc}[1]{\textcolor{red}{\textbf{Question:} #1}}


\newcommand{\transformationTheorem}[2]{%
  \begin{theorem}
    For every setting of the input parameters #1 to #2 such that the given preconditions
    hold, #2 raises an exception (at compile time or run time) or returns a valid transformation. A valid transformation has the following properties:
    \begin{enumerate}
        \item \textup{(Appropriate output domain).} 
        For every element $v$ in \texttt{input\_domain}, $\function(v)$ is in \texttt{output\_domain} or raises a data-independent runtime exception.
        
        \item \textup{(Domain-metric compatibility).} 
        The domain \texttt{input\_domain} matches one of the possible domains listed in the definition of \texttt{input\_metric}, 
        and likewise \texttt{output\_domain} matches one of the possible domains listed in the definition of \texttt{output\_metric}.
        
        \item \textup{(Stability guarantee).} 
        For every pair of elements $u, v$ in \texttt{input\_domain} and for every pair $(\din, \dout)$, 
        where \din\ has the associated type for \texttt{input\_metric} and \dout\ has the associated type for \\ \texttt{output\_metric}, 
        if $u, v$ are \din-close under \texttt{input\_metric} and $\texttt{stability\_map}(\din) \leq \dout$, 
        then $\function(u), \function(v)$ are $\dout$-close under \texttt{output\_metric}.
    \end{enumerate}
  \end{theorem}
}


\title{\texttt{fn sample\_discrete\_laplace}}
\author{Michael Shoemate}

\begin{document}
\maketitle

\contrib
Proves soundness of \texttt{fn sample\_discrete\_laplace} in \asOfCommit{mod.rs}{0be3ab3e6}.
This proof is an adaptation of \href{https://arxiv.org/pdf/2004.00010.pdf#subsection.5.2}{subsection 5.2} of \cite{CKS20}.

\section{Vetting history}
\begin{itemize}
    \item \vettingPR{519}
\end{itemize}

\section{Pseudocode}
\subsubsection*{Precondition}
$\texttt{scale} \in \mathbb{Q} \land \texttt{scale} > 0$

\subsubsection*{Implementation}        
\begin{lstlisting}[language=Python]
def sample_discrete_laplace(scale) -> int:
    if (scale == 0):
        return 0
        
    inv_scale = recip(scale)
    
    while True:
        sign = sample_standard_bernoulli()
        magnitude = sample_geometric_exp_fast(inv_scale)
        
        if sign or magnitude != 0:
            if sign:
                return -magnitude
            else:
                return magnitude
\end{lstlisting}

\subsubsection*{Postcondition}
\texttt{sample\_discrete\_laplace} returns an integer sample from $\mathcal{L}_\mathbb{Z}(0, scale)$, or an error propagated from \texttt{sample\_geometric\_exp\_fast} or \texttt{sample\_standard\_bernoulli}.

\section{Proof}

First, define the target distribution.

\begin{definition}
    (Discrete Laplace). Let $\mu, \sigma \in \mathbb{R}$ with $\sigma > 0$. The discrete laplace distribution with location $\mu$ and scale $s$ is denoted $\mathcal{L}_\mathbb{Z}(\mu, s)$. It is a probability distribution supported on the integers and defined by \cite{BV17}
\begin{equation*}
    \forall x \in \mathbb{Z} \quad  P[X = x] = \frac{e^{-1/s} - 1}{e^{-1/s} + 1} e^{-|x|/s} \quad \text{where } X \sim \mathcal{L}_\mathbb{Z}(\mu, s)
\end{equation*}
\end{definition}

% We must show that the return value of \texttt{sample\_discrete\_laplace}, denoted $y$, conditioned on not returning an error, is a sample from $\mathcal{L}_\mathbb{Z}(0, scale)$.

\begin{lemma}
\label{P_B_Y_ne_T_0}
Let $B \sim Bernoulli(1/2)$ and $Y \sim Geometric(1 - e^{-1/s})$ for some $s > 0$. Then $P[(B, Y) \neq (\top, 0)] = \frac{1}{2} (e^{-1/s} + 1)$\cite{CKS20}.
\end{lemma}

\begin{proof}
\begin{align*}
    P[(B, Y) \neq (\top, 0)] &= P[B = \top, Y > 0] + P[B = \bot] && \text{by LOTP} \\
    &= P[B = \top] P[Y > 0] + P[B = \bot] && \text{by independence of B, Y} \\
    &= \frac{1}{2} e^{-1/s} + \frac{1}{2} \\
    &= \frac{1}{2} (e^{-1/s} + 1)
\end{align*}
\end{proof}

\begin{theorem}
\label{P_Lx_BY_ne_T0}
Given random variables $B \sim Bernoulli(1/2)$ and $Y \sim Geometric(1 - e^{-1/s})$, define $X|_{B=\top} = Y$, and $X|_{B=\bot} = -Y$. If $(B, Y) \neq (\top, 0)$, then $X \sim \mathcal{L}_\mathbb{Z}(0, scale)$. That is, $P[X = x | (B, Y) \neq (\top, 0)] = \frac{1 - e^{-1/\sigma}}{1 + e^{-1/\sigma}} e^{-|x|/\sigma}$ for any $x \in \mathbb{Z}$\cite{CKS20}.
\end{theorem}

\begin{proof}
\begin{align*}
P[X = x | (B, Y) \neq (\top, 0)] &= \frac{P[X = x, (B, Y) \neq (\top, 0)]}{P[(B, Y) \neq (\top, 0)]} \\
    &= \frac{P[X = |x|, B = \mathbb{I}[x < 0]]}{P[(B, Y) \neq (\top, 0)]} && \text{since x = }\pm y \\
    &= \frac{P[X = |x|] P[B = \mathbb{I}[x < 0]]}{P[(B, Y) \neq (\top, 0)]} && \text{by independence of B, Y} \\
    &= \frac{P[X = |x|] \frac{1}{2}}{\frac{1}{2} (e^{-1/s} + 1)} && \text{by } \ref{P_B_Y_ne_T_0} \\
    &= \frac{1 - e^{-1/s}}{1 + e^{-1/s}} e^{-|x|/s} \\
\end{align*}
\end{proof}

Since \texttt{sign} is a realization of $B$, \texttt{magnitude} is a realization of $Y \sim Geometric(1 - e^{-1/scale})$, and the return is conditioned on $(B, Y) \neq (\top, 0)$ and not returning an error, then by \ref{P_Lx_BY_ne_T0} the function's return value is a realization of $\mathcal{L}_\mathbb{Z}(0, scale)$.


\bibliographystyle{alpha}
\bibliography{mod}

\end{document}