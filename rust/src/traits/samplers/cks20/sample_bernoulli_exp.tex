\documentclass{article}
\documentclass[11pt,letterpaper]{article}
\usepackage[top=1in, right=1in, left=1in, bottom=1.5in]{geometry}

\usepackage{amsmath,amsthm,amsfonts,amssymb,amscd}
\usepackage{listings}
\usepackage{hyperref}
\usepackage{xcolor}
\usepackage{xr}
\usepackage[outputdir=./out]{minted}

\hypersetup{
  colorlinks=true,
  linkcolor=blue,
  linkbordercolor={0 0 1}
}


% Pseudocode
\definecolor{codegreen}{rgb}{0,0.6,0}
\definecolor{codegray}{rgb}{0.5,0.5,0.5}
\definecolor{codepurple}{rgb}{0.58,0,0.82}
\definecolor{backcolour}{rgb}{0.95,0.95,0.92}

\setminted[python]{
    firstline=2,
    firstnumber=1,
    linenos=true,
    numbersep=5pt,
    gobble=0,
    frame=leftline,
    framerule=0.4pt,
    framesep=2mm,
    funcnamehighlighting=true,
    tabsize=4,
    obeytabs=false,
    mathescape=false
    showspaces=false,
    showtabs =false,
    texcomments=true,
    fontsize=\small
}
% Commands
\newcommand{\dependsOn}[2]{\immediate\write18{
    mkdir -p out_deps && 
    git show #1:#2 > ./out_deps/$(basename #2)
}}

\newcommand{\MultiSet}{\mathrm{MultiSet}}
\newcommand{\MultiSets}{\mathrm{MultiSets}}
\newcommand{\len}{\mathrm{len}}
\newcommand{\din}{\mathrm{d_{in}}}
\newcommand{\dout}{\mathrm{d_{out}}}
\newcommand{\Relation}{\mathrm{Relation}}
\newcommand{\question}[1]{\textcolor{red}{\textbf{Question:} #1}}
\newcommand{\ellOne}{\mathrm{\ell_1}}
\newcommand{\maxUsize}{\texttt{usize::MAX}}
\newcommand{\function}{\texttt{function}}
\newcommand{\True}{\texttt{True}}
\newcommand{\todo}[1]{{\begin{center} \textcolor{teal}{{\huge TODO:} #1}\end{center}}}
\newcommand{\notebig}[1]{{ \textcolor{red}{{\huge Note:} #1}}}

\newtheorem{theorem}{Theorem}[section]
\newtheorem{lemma}[theorem]{Lemma}

\theoremstyle{definition}
\newtheorem{remark}{Remark}
\newtheorem{definition}[theorem]{Definition}
\newtheorem{observation}{Observation}
\newtheorem{note}{Note}
\newtheorem{hope}{Hope}
\newtheorem{warning}{Warning}
\newtheorem{problem}{Problem}
\newtheorem{fear}{Fear}


\title{\texttt{fn sample\_bernoulli\_exp}}
\author{Michael Shoemate}

\begin{document}
\maketitle

From \href{https://arxiv.org/pdf/2004.00010.pdf#subsection.5.1}{subsection 5.1} of \cite{CKS20}.

\section{Pseudocode}
\subsubsection*{Precondition}
$\texttt{x} \in \mathbb{Q} \land \texttt{x} > 0$

\subsubsection*{Implementation}        
\begin{lstlisting}[language=Python]
def sample_bernoulli_exp(x) -> bool:
    while x >= 1:
        if sample_bernoulli_exp1(1):
            x -= 1
        else: 
            return False
    return sample_bernoulli_exp1(x)
\end{lstlisting}

\subsubsection*{Postcondition}
\texttt{sample\_bernoulli\_exp} returns a boolean sample from $Bernoulli(exp(-x))$, or an error propagated from \texttt{sample\_bernoulli\_exp1}.

\section{Proof}

Let $r$ be the return value of \texttt{sample\_bernoulli\_exp(x)}, assuming the preconditions are met, and the function does not return an error.

\begin{theorem}
$r$ is a realization of $R \sim Bernoulli(exp(-x))$, where $x \in \mathbb{Q}$ and $x > 0$ \cite{CKS20}.
\end{theorem}

\begin{proof}
Let $B_i \sim Bernoulli(exp(-1))$ be the sequence of booleans returned from \texttt{sample\_bernoulli\_exp1(1)}, and $C \sim Bernoulli(exp(-(x - \lfloor x \rfloor)))$.
The function only returns $\top$ if $B_i = \top$ for all $i$ and $C = \top$.

\begin{align*}
    P[R = \top] &= P[B_1 = B_2 = ... = B_{\lfloor x \rfloor} = C = \top] \\
    &= \prod_{i=1}^{\lfloor x \rfloor} P[B_i = \top] P[C = \top] && \text{All $B_i$ are independent.} \\
    &= exp(-1)^{\lfloor x \rfloor} exp(\lfloor x \rfloor - x) \\
    &= exp(-x)
\end{align*}
\end{proof}

\bibliographystyle{alpha}
\bibliography{mod}

\end{document}