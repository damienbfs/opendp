\documentclass{article}
% common styling and macros shared by all proof files

\usepackage[top=1in, right=1in, left=1in, bottom=1.5in]{geometry}

\usepackage{amsmath,amsthm,amsfonts,amssymb,amscd}
\usepackage{listings}
\usepackage{hyperref}
\usepackage{xcolor}
\usepackage{xr}

\usepackage{enumerate} 
\usepackage{physics}
\usepackage{fancyhdr}
\usepackage{hyperref}
\usepackage{graphicx}
\usepackage{tcolorbox}
\usepackage{catchfile}
\usepackage{pdftexcmds}
\usepackage[T1]{fontenc}

% hyperref
\hypersetup{
  colorlinks=true,
  linkcolor=blue,
  linkbordercolor={0 0 1}
}

% \contrib macro to indicate inclusion in "contrib".
\usepackage{tcolorbox}
\newtcolorbox{contrib_box}{colback=red!5!white,colframe=red!75!black}
\newcommand{\contrib}{{\begin{contrib_box}This proof resides in \textbf{``contrib''} because it has not completed the vetting process.\end{contrib_box}}} 

% asOfCommit macro to version a code dependency. Arguments:
%    #1: relative path to file you are dependent on
%    #2: commit hash it was last edited. If outdated, this should be the second hash in the footnoote. Otherwise,
%            git log -n 1 --pretty=format:%h -- path/to/file.rs
\makeatletter
\ifnum\pdf@shellescape=1
   % "private" command that builds a link to a blob
  \newcommand{\linkOpendpBlob}[3]{%
    \href{https://github.com/opendp/opendp/blob/#1/#2#3}{\path{#3} at commit #1}}

  % latex macro expansion has a separate phase for \input evaluation
  %     immediately evaluate a command to write a temp file to ./out containing the current directory
  \immediate\write18{[ ! -f out/cwd.txt ] && (mkdir -p out && git rev-parse --show-prefix | sed "s|_|\@backslashchar\@backslashchar\@backslashchar_|g" > out/cwd.txt)}
  %     ...and then retrieve the current working directory by loading the temp file
  \CatchFileDef\GitWorkingDir{out/cwd.txt}{\endlinechar=-1}

  % command for building the (up to date) or (outdated) status
  \newcommand{\fileStatus}[2]{%
  \setbox0=\hbox{\input|"git --no-pager log -n1 --pretty='\@percentchar H' #1 | grep -E '^#2.*'"\unskip}\ifdim\wd0=0pt
        (outdated\footnote{See new changes with \texttt{git diff #2..\input|"git --no-pager log -n1 --pretty='\@percentchar h' #1" \GitWorkingDir\path{#1}}})\else
        (up to date)\fi
  }

  \newcommand{\asOfCommit}[2]{%
      % permalink the target
      \linkOpendpBlob{#2}{\GitWorkingDir}{#1}
      % conditionally add (outdated) or (up to date) depending on matching commit hash
      \fileStatus{#1}{#2}%
  }
\else
  % simplified command if shell-escape not enabled
  \newcommand{\asOfCommit}[2]{#1 at commit #2 (unknown status\footnote{Shell-escape is not enabled. Enable \texttt{--shell-escape} to check if this proof is up-to-date with the code.})}
\fi
\makeatother

% \vettingPR macro to link a PR. Arguments:
%    #1: PR number
\newcommand{\vettingPR}[1]{\href{https://github.com/opendp/opendp/pull/#1}{Pull Request \##1}}

% \OpenDPVersion macro to get library version. Mainly used for \rustdoc
\makeatletter
\ifnum\pdf@shellescape=1
  % latex macro expansion has a separate phase for \input evaluation
  %     immediately evaluate a command to write a temp file to ./out containing the current version
  \immediate\write18{[ ! -f out/version.txt ] && (mkdir -p out && head -n 1 `git rev-parse --show-toplevel`/VERSION | sed "s|0.0.0+development|latest|g" > out/version.txt)}
  %     ...and then retrieve the version by loading the temp file
  \CatchFileDef\OpenDPVersion{out/version.txt}{\endlinechar=-1}
\else
  % if shell commands are not enabled, just claim latest
  \newcommand{\OpenDPVersion}{latest}
\fi
\makeatother

% for links to rustdoc items in OpenDP. Arguments:
%    #1: path to item, and designation as trait, struct, fn, etc.
%    #2: item name
\newcommand{\rustdoc}[2]{\href{https://docs.rs/opendp/\OpenDPVersion/opendp/#1.#2.html}{\texttt{#2}}}

% for links to external dependencies. Arguments:
%    #1: crate name
%    #2: path to item, and designation as trait, struct, fn, etc.
%    #3: item name
\newcommand{\docsrs}[3]{\href{https://docs.rs/#1/latest/#1/#2.#3.html}{\texttt{#3}}}

% minted (pseudocode)
\definecolor{codegreen}{rgb}{0,0.6,0}
\definecolor{codegray}{rgb}{0.5,0.5,0.5}
\definecolor{codepurple}{rgb}{0.58,0,0.82}
\definecolor{backcolour}{rgb}{0.95,0.95,0.92}

\lstdefinestyle{mystyle}{
    backgroundcolor=\color{backcolour},   
    commentstyle=\color{codegreen},
    keywordstyle=\color{magenta},
    numberstyle=\tiny\color{codegray},
    stringstyle=\color{codepurple},
    basicstyle=\ttfamily\footnotesize,
    breakatwhitespace=false,         
    breaklines=true,                 
    captionpos=b,                    
    keepspaces=true,                 
    numbers=left,                    
    numbersep=5pt,                  
    showspaces=false,                
    showstringspaces=false,
    showtabs=false,                  
    tabsize=2
}

\lstset{style=mystyle}

% common commands
\theoremstyle{definition}
\newtheorem{theorem}{Theorem}[section]
\newtheorem{lemma}[theorem]{Lemma}
\newtheorem{definition}[theorem]{Definition}
\newtheorem{warning}{Warning}
\newtheorem{corollary}{Corollary}
\newtheorem{proposition}{Proposition}
\newtheorem{remark}{Remark}
\newtheorem{observation}{Observation}
\newtheorem{note}{Note}

\newcommand{\vicki}[1]{{ {\color{olive}{(vicki)~#1}}}}
\newcommand{\hanwen}[1]{{ {\color{purple}{(hanwen)~#1}}}}
\newcommand{\zach}[1]{{ {\color{red}{(zach)~#1}}}}

\newcommand{\MultiSet}{\mathrm{MultiSet}}
\newcommand{\len}{\mathrm{len}}
\newcommand{\din}{\texttt{d\_in}}
\newcommand{\dout}{\texttt{d\_out}}
\newcommand{\T}{\texttt{T} }
\newcommand{\F}{\texttt{F} }
\newcommand{\Map}{\texttt{Map}}
\newcommand{\X}{\mathcal{X}}
\newcommand{\Y}{\mathcal{Y}}
\newcommand{\True}{\texttt{True}}
\newcommand{\False}{\texttt{False}}
\newcommand{\clamp}{\texttt{clamp}}
\newcommand{\function}{\texttt{function}}
\newcommand{\float}{\texttt{float }}
\newcommand{\questionc}[1]{\textcolor{red}{\textbf{Question:} #1}}


\newcommand{\transformationTheorem}[2]{%
  \begin{theorem}
    For every setting of the input parameters #1 to #2 such that the given preconditions
    hold, #2 raises an exception (at compile time or run time) or returns a valid transformation. A valid transformation has the following properties:
    \begin{enumerate}
        \item \textup{(Appropriate output domain).} 
        For every element $v$ in \texttt{input\_domain}, $\function(v)$ is in \texttt{output\_domain} or raises a data-independent runtime exception.
        
        \item \textup{(Domain-metric compatibility).} 
        The domain \texttt{input\_domain} matches one of the possible domains listed in the definition of \texttt{input\_metric}, 
        and likewise \texttt{output\_domain} matches one of the possible domains listed in the definition of \texttt{output\_metric}.
        
        \item \textup{(Stability guarantee).} 
        For every pair of elements $u, v$ in \texttt{input\_domain} and for every pair $(\din, \dout)$, 
        where \din\ has the associated type for \texttt{input\_metric} and \dout\ has the associated type for \\ \texttt{output\_metric}, 
        if $u, v$ are \din-close under \texttt{input\_metric} and $\texttt{stability\_map}(\din) \leq \dout$, 
        then $\function(u), \function(v)$ are $\dout$-close under \texttt{output\_metric}.
    \end{enumerate}
  \end{theorem}
}


\title{\texttt{trait SampleUniformIntBelow}}
\author{Michael Shoemate}

\begin{document}
\maketitle

\contrib

\subsection*{PR History}
\begin{itemize}
    \item \vettingPR{473}
\end{itemize}

This document proves that the implementations of \rustdoc{traits/samplers/uniform/trait}{SampleUniformIntBelow} in \asOfCommit{mod.rs}{f5bb719} 
satisfy the definition of the \texttt{SampleUniformIntBelow} trait.

\begin{definition}
    \label{sample-uniform-int-below}

    The \texttt{SampleUniformIntBelow} trait defines a function \texttt{sample\_uniform\_int\_below}.

    For any setting of the input parameter \texttt{upper},
    \texttt{sample\_uniform\_int\_below} either
    \begin{itemize}
        \item raises an exception if there is a lack of system entropy,
        \item returns \texttt{out} where \texttt{out} is uniformly distributed between $[0, upper)$.
    \end{itemize}

    If \texttt{trials} is specified, the function will attempt to sample uniformly at random \texttt{trials} times,
    and will raise an exception if it fails to do so.
    If \texttt{trials} is not specified, the function will attempt to sample uniformly at random indefinitely.
    Setting \texttt{trials} causes the function to run in constant time.
\end{definition}

There are two \texttt{impl}'s (implementations): one for unsigned integers, and one for big integers.
To prove correctness of each \texttt{impl}, we prove correctness of the implementation of \texttt{sample\_uniform\_int\_below}.

\tableofcontents

\section{\texttt{impl} for Unsigned Integers}
This corresponds to \texttt{impl SampleUniformIntBelow for \$ty} in Rust.
\texttt{sample\_uniform\_int\_below} uses rejection sampling. 
In each round all bits of the integer are filled randomly, drawing an unsigned integer uniformly at random.
The algorithm returns the sample, modulo the upper bound, so long as the sample is not one of the final "div" largest integers.

\subsection{Hoare Triple}
\subsubsection*{Preconditions}
\begin{itemize}
    \item \textbf{User-specified types:}
    \begin{itemize}
        \item Variable \texttt{upper} must be of type \texttt{T}
        \item Variable \texttt{trials} is optional of type \texttt{int}, and is non-negative
        \item Type \texttt{T} is the type the trait is implemented for (one of u8, u16, u32, u64, u128, usize)
    \end{itemize}
\end{itemize}

\subsubsection*{Pseudocode}

\lstinputlisting[language=Python,firstline=2]{./pseudocode/sample_uniform_int_below_native.py}

\subsubsection*{Postcondition}
The postcondition is supplied by \ref{sample-uniform-int-below}.

\subsection{Proof}
\begin{proof} 
\label{unsigned-integer-proof}
Assuming that \texttt{T.sample\_uniform\_int()} is correctly implemented,
then \texttt{v} is a sample between zero and \texttt{T.MAX} inclusive, the greatest representable number of type \texttt{T}.

You could sample one of \texttt{upper} values uniformly at random by rejecting \texttt{v} if it is larger than \texttt{upper}.
That is, only return \texttt{v} if \texttt{v} is less than \texttt{upper}.

It is equivalent to extend the acceptance region, 
by returning \texttt{v \% 2} if \texttt{v} is less than \texttt{upper * 2}, 
so long as \texttt{upper * 2 <= T.MAX}.
This reduces the rejection rate, which increases algorithm performance.

There are \texttt{T.MAX \% upper} remaining elements if you were to 
extend the acceptance region to the greatest multiple of \texttt{upper} that is less than \texttt{T.MAX}.
Therefore conditioning \texttt{v} on being less than \texttt{T.MAX - T.MAX \% upper} 
results in \texttt{v \% upper} being an unbiased, uniformly distributed sample.

When \texttt{trials} is specified, the algorithm will attempt to sample uniformly at random \texttt{trials} times,
and will raise an exception if it fails to do so.
Only the first successful sample is kept.

\noindent Therefore, for any value of \texttt{upper}, the function satisfies the postcondition.
\end{proof}


\section{\texttt{impl} for Big Integers}
This corresponds to \texttt{impl SampleUniformIntBelow for UBig} in Rust.
This algorithm uses the same algorithm and argument as used for unsigned native integers, 
but this time the bit depth is dynamically chosen to fill the last byte of a series of bytes long enough to hold \texttt{upper}.

\subsection{Hoare Triple}
\subsubsection*{Preconditions}
\begin{itemize}
    \item \textbf{User-specified types:}
    \begin{itemize}
        \item Variable \texttt{upper} must be of type \texttt{UBig}
        \item Variable \texttt{trials} is optional of type \texttt{int}, and is non-negative
    \end{itemize}
\end{itemize}

\subsubsection*{Pseudocode}

\lstinputlisting[language=Python,firstline=2]{./pseudocode/sample_uniform_int_below_ubig.py}

\subsubsection*{Postcondition}
The postcondition is supplied by \ref{sample-uniform-int-below}.

\subsection{Proof}
\begin{proof} 

\texttt{byte\_len} is the fewest number of bytes necessary to represent \texttt{upper}, 
which is $ceil(ceil(log_2(upper)) / 8)$.

This proof follows the same logic as in \ref{unsigned-integer-proof},
but the constants are generalized.
\texttt{max} is the largest representable number in \texttt{byte\_len} bytes, corresponding to \texttt{T.MAX}.
\texttt{v} is an integer sampled uniformly below \texttt{max} by randomly filling bits with bernoulli samples.
The algorithm terminates when \texttt{v} is below the same threshold.

When \texttt{trials} is specified, the algorithm will attempt to sample uniformly at random \texttt{trials} times,
and will raise an exception if it fails to do so.
Only the first successful sample is kept.

\noindent Therefore, for any value of \texttt{upper}, the function satisfies the postcondition.
\end{proof}

\end{document}
