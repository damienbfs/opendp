\documentclass[11pt,letterpaper]{article}
\usepackage[top=1in, right=1in, left=1in, bottom=1.5in]{geometry}

\usepackage{amsmath,amsthm,amsfonts,amssymb,amscd}
\usepackage{listings}
\usepackage{hyperref}
\usepackage{xcolor}
\usepackage{xr}
\usepackage[outputdir=./out]{minted}

\hypersetup{
  colorlinks=true,
  linkcolor=blue,
  linkbordercolor={0 0 1}
}


% Pseudocode
\definecolor{codegreen}{rgb}{0,0.6,0}
\definecolor{codegray}{rgb}{0.5,0.5,0.5}
\definecolor{codepurple}{rgb}{0.58,0,0.82}
\definecolor{backcolour}{rgb}{0.95,0.95,0.92}

\setminted[python]{
    firstline=2,
    firstnumber=1,
    linenos=true,
    numbersep=5pt,
    gobble=0,
    frame=leftline,
    framerule=0.4pt,
    framesep=2mm,
    funcnamehighlighting=true,
    tabsize=4,
    obeytabs=false,
    mathescape=false
    showspaces=false,
    showtabs =false,
    texcomments=true,
    fontsize=\small
}
% Commands
\newcommand{\dependsOn}[2]{\immediate\write18{
    mkdir -p out_deps && 
    git show #1:#2 > ./out_deps/$(basename #2)
}}

\newcommand{\MultiSet}{\mathrm{MultiSet}}
\newcommand{\MultiSets}{\mathrm{MultiSets}}
\newcommand{\len}{\mathrm{len}}
\newcommand{\din}{\mathrm{d_{in}}}
\newcommand{\dout}{\mathrm{d_{out}}}
\newcommand{\Relation}{\mathrm{Relation}}
\newcommand{\question}[1]{\textcolor{red}{\textbf{Question:} #1}}
\newcommand{\ellOne}{\mathrm{\ell_1}}
\newcommand{\maxUsize}{\texttt{usize::MAX}}
\newcommand{\function}{\texttt{function}}
\newcommand{\True}{\texttt{True}}
\newcommand{\todo}[1]{{\begin{center} \textcolor{teal}{{\huge TODO:} #1}\end{center}}}
\newcommand{\notebig}[1]{{ \textcolor{red}{{\huge Note:} #1}}}

\newtheorem{theorem}{Theorem}[section]
\newtheorem{lemma}[theorem]{Lemma}

\theoremstyle{definition}
\newtheorem{remark}{Remark}
\newtheorem{definition}[theorem]{Definition}
\newtheorem{observation}{Observation}
\newtheorem{note}{Note}
\newtheorem{hope}{Hope}
\newtheorem{warning}{Warning}
\newtheorem{problem}{Problem}
\newtheorem{fear}{Fear}


\title{OpenDP \texttt{make\_base\_laplace}}
\author{security@opendp.org}

\dependsOn{HEAD}{../../definitions.tex}


\begin{document}

\maketitle

The discrete laplace measurement adds an approximation to laplacian noise to an input argument.
We can't just add laplacian noise directly, because it doesn't admit $\epsilon$ differential privacy.

Instead use a discrete version of laplacian noise, sampled from the two-sided geometric distribution.
To use this distribution, must first scale up the inputs to be represented in an integer space.

We essentially discretize the set of real numbers into integers, noise the integers, and then scale back to the real number line (floats).

To transform into the integer space, we need to rescale the inputs by some factor c, and then cast to the nearest integer.
We track the stability of these transforms indirectly through the $\texttt{make\_lipschitz\_extension}$ and $\texttt{make\_integerize}$ transformation constructors.
The $\texttt{make\_base\_discrete\_laplace}$ is simply a chain of existing transformations, measurements and postprocessing.


\section{Pseudocode}
\label{sec:python-pseudocode}

\inputminted{python}{pseudocode.py}

\section{Proof: Privacy Relation}

\end{document}
