\documentclass{article}
% common styling and macros shared by all proof files

\usepackage[top=1in, right=1in, left=1in, bottom=1.5in]{geometry}

\usepackage{amsmath,amsthm,amsfonts,amssymb,amscd}
\usepackage{listings}
\usepackage{hyperref}
\usepackage{xcolor}
\usepackage{xr}

\usepackage{enumerate} 
\usepackage{physics}
\usepackage{fancyhdr}
\usepackage{hyperref}
\usepackage{graphicx}
\usepackage{tcolorbox}
\usepackage{catchfile}
\usepackage{pdftexcmds}
\usepackage[T1]{fontenc}

% hyperref
\hypersetup{
  colorlinks=true,
  linkcolor=blue,
  linkbordercolor={0 0 1}
}

% \contrib macro to indicate inclusion in "contrib".
\usepackage{tcolorbox}
\newtcolorbox{contrib_box}{colback=red!5!white,colframe=red!75!black}
\newcommand{\contrib}{{\begin{contrib_box}This proof resides in \textbf{``contrib''} because it has not completed the vetting process.\end{contrib_box}}} 

% asOfCommit macro to version a code dependency. Arguments:
%    #1: relative path to file you are dependent on
%    #2: commit hash it was last edited. If outdated, this should be the second hash in the footnoote. Otherwise,
%            git log -n 1 --pretty=format:%h -- path/to/file.rs
\makeatletter
\ifnum\pdf@shellescape=1
   % "private" command that builds a link to a blob
  \newcommand{\linkOpendpBlob}[3]{%
    \href{https://github.com/opendp/opendp/blob/#1/#2#3}{\path{#3} at commit #1}}

  % latex macro expansion has a separate phase for \input evaluation
  %     immediately evaluate a command to write a temp file to ./out containing the current directory
  \immediate\write18{[ ! -f out/cwd.txt ] && (mkdir -p out && git rev-parse --show-prefix | sed "s|_|\@backslashchar\@backslashchar\@backslashchar_|g" > out/cwd.txt)}
  %     ...and then retrieve the current working directory by loading the temp file
  \CatchFileDef\GitWorkingDir{out/cwd.txt}{\endlinechar=-1}

  % command for building the (up to date) or (outdated) status
  \newcommand{\fileStatus}[2]{%
  \setbox0=\hbox{\input|"git --no-pager log -n1 --pretty='\@percentchar H' #1 | grep -E '^#2.*'"\unskip}\ifdim\wd0=0pt
        (outdated\footnote{See new changes with \texttt{git diff #2..\input|"git --no-pager log -n1 --pretty='\@percentchar h' #1" \GitWorkingDir\path{#1}}})\else
        (up to date)\fi
  }

  \newcommand{\asOfCommit}[2]{%
      % permalink the target
      \linkOpendpBlob{#2}{\GitWorkingDir}{#1}
      % conditionally add (outdated) or (up to date) depending on matching commit hash
      \fileStatus{#1}{#2}%
  }
\else
  % simplified command if shell-escape not enabled
  \newcommand{\asOfCommit}[2]{#1 at commit #2 (unknown status\footnote{Shell-escape is not enabled. Enable \texttt{--shell-escape} to check if this proof is up-to-date with the code.})}
\fi
\makeatother

% \vettingPR macro to link a PR. Arguments:
%    #1: PR number
\newcommand{\vettingPR}[1]{\href{https://github.com/opendp/opendp/pull/#1}{Pull Request \##1}}

% \OpenDPVersion macro to get library version. Mainly used for \rustdoc
\makeatletter
\ifnum\pdf@shellescape=1
  % latex macro expansion has a separate phase for \input evaluation
  %     immediately evaluate a command to write a temp file to ./out containing the current version
  \immediate\write18{[ ! -f out/version.txt ] && (mkdir -p out && head -n 1 `git rev-parse --show-toplevel`/VERSION | sed "s|0.0.0+development|latest|g" > out/version.txt)}
  %     ...and then retrieve the version by loading the temp file
  \CatchFileDef\OpenDPVersion{out/version.txt}{\endlinechar=-1}
\else
  % if shell commands are not enabled, just claim latest
  \newcommand{\OpenDPVersion}{latest}
\fi
\makeatother

% for links to rustdoc items in OpenDP. Arguments:
%    #1: path to item, and designation as trait, struct, fn, etc.
%    #2: item name
\newcommand{\rustdoc}[2]{\href{https://docs.rs/opendp/\OpenDPVersion/opendp/#1.#2.html}{\texttt{#2}}}

% for links to external dependencies. Arguments:
%    #1: crate name
%    #2: path to item, and designation as trait, struct, fn, etc.
%    #3: item name
\newcommand{\docsrs}[3]{\href{https://docs.rs/#1/latest/#1/#2.#3.html}{\texttt{#3}}}

% minted (pseudocode)
\definecolor{codegreen}{rgb}{0,0.6,0}
\definecolor{codegray}{rgb}{0.5,0.5,0.5}
\definecolor{codepurple}{rgb}{0.58,0,0.82}
\definecolor{backcolour}{rgb}{0.95,0.95,0.92}

\lstdefinestyle{mystyle}{
    backgroundcolor=\color{backcolour},   
    commentstyle=\color{codegreen},
    keywordstyle=\color{magenta},
    numberstyle=\tiny\color{codegray},
    stringstyle=\color{codepurple},
    basicstyle=\ttfamily\footnotesize,
    breakatwhitespace=false,         
    breaklines=true,                 
    captionpos=b,                    
    keepspaces=true,                 
    numbers=left,                    
    numbersep=5pt,                  
    showspaces=false,                
    showstringspaces=false,
    showtabs=false,                  
    tabsize=2
}

\lstset{style=mystyle}

% common commands
\theoremstyle{definition}
\newtheorem{theorem}{Theorem}[section]
\newtheorem{lemma}[theorem]{Lemma}
\newtheorem{definition}[theorem]{Definition}
\newtheorem{warning}{Warning}
\newtheorem{corollary}{Corollary}
\newtheorem{proposition}{Proposition}
\newtheorem{remark}{Remark}
\newtheorem{observation}{Observation}
\newtheorem{note}{Note}

\newcommand{\vicki}[1]{{ {\color{olive}{(vicki)~#1}}}}
\newcommand{\hanwen}[1]{{ {\color{purple}{(hanwen)~#1}}}}
\newcommand{\zach}[1]{{ {\color{red}{(zach)~#1}}}}

\newcommand{\MultiSet}{\mathrm{MultiSet}}
\newcommand{\len}{\mathrm{len}}
\newcommand{\din}{\texttt{d\_in}}
\newcommand{\dout}{\texttt{d\_out}}
\newcommand{\T}{\texttt{T} }
\newcommand{\F}{\texttt{F} }
\newcommand{\Map}{\texttt{Map}}
\newcommand{\X}{\mathcal{X}}
\newcommand{\Y}{\mathcal{Y}}
\newcommand{\True}{\texttt{True}}
\newcommand{\False}{\texttt{False}}
\newcommand{\clamp}{\texttt{clamp}}
\newcommand{\function}{\texttt{function}}
\newcommand{\float}{\texttt{float }}
\newcommand{\questionc}[1]{\textcolor{red}{\textbf{Question:} #1}}


\newcommand{\transformationTheorem}[2]{%
  \begin{theorem}
    For every setting of the input parameters #1 to #2 such that the given preconditions
    hold, #2 raises an exception (at compile time or run time) or returns a valid transformation. A valid transformation has the following properties:
    \begin{enumerate}
        \item \textup{(Appropriate output domain).} 
        For every element $v$ in \texttt{input\_domain}, $\function(v)$ is in \texttt{output\_domain} or raises a data-independent runtime exception.
        
        \item \textup{(Domain-metric compatibility).} 
        The domain \texttt{input\_domain} matches one of the possible domains listed in the definition of \texttt{input\_metric}, 
        and likewise \texttt{output\_domain} matches one of the possible domains listed in the definition of \texttt{output\_metric}.
        
        \item \textup{(Stability guarantee).} 
        For every pair of elements $u, v$ in \texttt{input\_domain} and for every pair $(\din, \dout)$, 
        where \din\ has the associated type for \texttt{input\_metric} and \dout\ has the associated type for \\ \texttt{output\_metric}, 
        if $u, v$ are \din-close under \texttt{input\_metric} and $\texttt{stability\_map}(\din) \leq \dout$, 
        then $\function(u), \function(v)$ are $\dout$-close under \texttt{output\_metric}.
    \end{enumerate}
  \end{theorem}
}


\usepackage[
        backend=biber,
        style=authoryear-comp,
        sorting=nyt,
    ]{biblatex}
\addbibresource{ref.bib} 

\usepackage{thmtools}
\usepackage{thm-restate}
\newtheorem{thm}{Theorem}
\newtheorem{defn}{Definition} 
\newtheorem{prop}{Proposition}
\newcommand{\mrm}[1]{\mathrm{#1}}


\title{\texttt{fn make\_tulap}}
\author{Yu-Ju Ku, Jordan Awan}
\begin{document}

\maketitle

\contrib

Proves soundness of \rustdoc{measurements/fn}{make\_tulap} in 
% \asOfCommit{mod.rs}{f5bb719}.

\texttt{make\_tulap} accepts a parameter \texttt{m} of type \texttt{float}, which is a rational number, a parameter \texttt{b} of type \texttt{float}, which is greater than 0 and less than 1, and a parameter \texttt{q} of type \texttt{float}, which is greater than or equal to 0 and less than 1.
The function on the resulting measurement takes in float data points that follow Binomial Distribution \texttt{arg} and returns the probability \texttt{prob}, or the complement $\texttt{!arg}$ with probability $1 - \texttt{prob}$.

% \begin{tcolorbox}
%     \begin{warning}[Code is not constant-time]
%         \texttt{make\_randomized\_response\_bool} takes in a boolean \texttt{constant\_time} parameter that protects against timing attacks on the Bernoulli sampling procedure. 
%         However, the current implementation does not guard against other types of timing side-channels that can break differential privacy, e.g., non-constant time code execution due to branching.
%     \end{warning}
% \end{tcolorbox}

\subsection*{PR History}
\begin{itemize}
    \item \vettingPR{}
\end{itemize}

\section{Hoare Triple}

\subsection*{Preconditions}
\begin{itemize}
    % \item Variable \texttt{m} must be of type \texttt{float}
    \item Variable \texttt{b} must be of type \texttt{float}
    % \item Variable \texttt{q} must be of type \texttt{float}
    \item Variable \texttt{scale} must be of type \texttt{float}
    \item Type \texttt{float} must have trait \rustdoc{traits/trait}{SampleDiscreteLaplaceZ2k}
\end{itemize}

\subsection*{Pseudocode}

% def make_tulap(m: float, b: float, q: float, s: float):
%     input_domain = AllDomain(float)
%     input_metric = AbsoluteDistance(float)
%     output_measure = SmoothedMaxDivergence(float)

%     def privacy_map(d_in: float) -> float: |\label{line:map}|
%         if d_in == 0 or d_out > s:
%             return 0
%         else: 
%             return d_out
            
%     def qCND(u, f, c):     # CND quantile function for f
%         if u < c:
%             return qCND(1 - f(u), f, c) - 1
%         elif c <= u <= 1-c:
%             return (u - 1/2)/(1 - 2*c)
%         else:
%             return qCND(f(1-u),f ,c) + 1    
    
%     def function(arg: float) -> float: |\label{line:fn}|
    
%         prob = b
%         samples = m + sample_discrete_laplace_Z2k(prob) + sample_standard_uniform(prob)-1/2 
        
%         return Measurement(input_domain, function, input_metric, output_measure, privacy_map)

\begin{lstlisting}[language=Python, escapechar=|]
def make_tulap(b: float, scale: float):
    input_domain = AllDomain(float)
    input_metric = AbsoluteDistance(float)
    output_measure = SmoothedMaxDivergence(float)

    def privacy_map(d_in: float) -> float: |\label{line:map}|
        if d_in == 0 or d_out > scale:  # check that delta <= scale
            return 0
        else: 
            return d_out
            
    def qCND(u, f, c):         # CND quantile function for f
        if u < c:
            return qCND(1 - f(u), f, c) - 1
        elif c <= u <= 1-c: # the linear function 
            return (u - 1/2)/(1 - 2*c)
        else:
            return qCND(f(1-u),f ,c) + 1  
            
    def function(epsilon: float, delta: float) -> float: |\label{line:fn}|

        # inverse transform sampling of Tulap
        prob = b
        unif = sample_standard_uniform(prob)
        c = (1 - delta) / (1 + exp(epsilon))
        f = max(0, 1 - delta - exp(epsilon) * unif, exp(-epsilon) * (1 - delta - unif))
        samples = qCND(unif, f, c)
        return samples
        
    return Measurement(input_domain, function, input_metric, output_measure, privacy_map)
\end{lstlisting}

% samples = m + sample_discrete_laplace_Z2k(prob) + sample_standard_uniform(prob)-1/2 


\subsection*{Postcondition}

\validMeasurement{\texttt{(prob, m, b, q, QO)}}{\\ \texttt{make\_tulap}}

\section{Proof}

\begin{proof} 
\hfill
\begin{enumerate}
    \item \textbf{(Domain-metric compatibility.)} For \texttt{make\_tulap}, this corresponds to showing \texttt{AllDomain(bool)} is compatible with \texttt{DiscreteMetric}. 
    This follows directly from the definition of \rustdoc{metrics/struct}{DiscreteMetric}.

    \item \textbf{(Privacy guarantee.)} 

    \begin{tcolorbox}
\begin{note}[Proof relies on correctness of Discrete Laplace sampler and Uniform sampler]
The following proof makes use of the following lemmas that asserts the correctness of the Discrete Laplace sampler and Uniform sampler function.
    \begin{lemma}
    If system entropy is insufficient, \texttt{sample\_discrete\_laplace\_Z2k} raises an error. 
    Otherwise, \texttt{sample\_discrete\_laplace\_Z2k(shift, scale, k)}, the Discrete Laplace sampler function used in \texttt{make\_tulap}, 
    returns sample from the discrete laplace distribution on $\mathbb{Z} \cdot 2^k$.
    \end{lemma}
    \begin{lemma}
    If system entropy is insufficient, \texttt{sample\_standard\_uniform} raises an error. 
    Otherwise, \texttt{sample\_standard\_uniform(constant\_time)}, the Uniform sampler function used in \texttt{make\_tulap}, 
    returns sample drawn from Uniform[0,1).
    \end{lemma}
\end{note}
\end{tcolorbox}

    \texttt{sample\_discrete\_laplace\_Z2k} and \texttt{sample\_standard\_uniform} can only fail when the OpenSSL pseudorandom byte generator used in its implementation fails due to lack of system entropy. 
    This is usually related to the computer's physical environment and not the dataset. 
    The rest of this proof is conditioned on the assumption that \texttt{sample\_discrete\_laplace\_Z2k} and \texttt{sample\_standard\_uniform} does not raise an exception. \\
    The concept canonical noise distribution (CND)\parencite{awan2023canonical}, which captures whether a real-valued distribution is perfectly tailored to satisfy $f$-DP\parencite{dong2019gaussian}. We formalize this in Definition \ref{def2}\parencite{awan2023canonical}. 
    \begin{defn}\label{def2}  % Def 3.1
    Let $f$ be a symmetric nontrivial tradeoff function. A {continuous} distribution function $F$ is a \emph{canonical noise distribution} (CND) for $f$ if 
    \begin{enumerate}[(1)]
    \item for every statistic $S:\mscr X^n\rightarrow \RR$ with sensitivity $\Delta>0$, and $N\sim F(\cdot)$, the mechanism $S(X) + \Delta N$ satisfies $f$-DP. Equivalently, for every $m\in [0,1]$, $T(F(\cdot),F(\cdot-m))\geq f$,
    \item $f(\alpha)=T(F(\cdot),F(\cdot-1))(\alpha)$ for all $\alpha \in (0,1)$,
    \item $T(F(\cdot),F(\cdot-1))(\alpha) = F(F^{-1}(1-\alpha)-1)$ for all $\alpha \in (0,1)$,
    \item $F(x) = 1-F(-x)$ for all $x\in \RR$; that is, $F$ is the cdf of a random variable which is symmetric about zero.
    \end{enumerate}
    \end{defn}
    \begin{defn} \label{def1}  % Def 3.7
    Let $f$ be a symmetric nontrivial tradeoff function, and let {$c\in [0,1/2)$} be the unique fixed point of $f$: $f(c)=c$. We define $F_f:\mathbb{R}\rightarrow \mathbb{R}$ as  \[ F_f(x) = \begin{cases}
    f(1-F_f(x+1))&x<-1/2\\
    c(1/2-x) + (1-c)(x+1/2)&-1/2\leq x\leq 1/2\\
    1-f(F_f(x-1))&x>1/2.\\
    \end{cases}\]
    \end{defn}
    In Definition \ref{def1} \parencite{awan2023canonical}, the fact that there is a unique fixed point follows from the fact that $f$ is convex and decreasing, and so intersects the line $y=\alpha$ at a unique value. Note that in Definition \ref{def1} \parencite{awan2023canonical}, the cumulative distribution function (CDF) corresponds to a uniform random variable on the interval $[-1/2,1/2]$. However, due to the recursive nature of $F_f$ and the fact that $f$ is generally non-linear, the Canonical Noise Distribution (CND) from Definition \ref{def1} \parencite{awan2023canonical} need not be uniformly distributed on any other intervals.

    Theorem \ref{thm1} \parencite{awan2023canonical} below states that for any nontrivial tradeoff function, this construction yields a CND, which can be constructed as in Definition \ref{def1} \parencite{awan2023canonical}. This CND can be used to add perfectly calibrated noise to a statistic to achieve $f$-DP \parencite{dong2019gaussian}. 
    
    \begin{thm} \label{thm1}  % Thm 3.9
    Let $f$ be a symmetric nontrivial tradeoff function and let $F_f$ be as in Definition \ref{def1} \parencite{awan2023canonical}. Then $F_f$ is a canonical noise distribution for $f$. 
    \end{thm}
    
    In Definition \ref{def1} \parencite{awan2023canonical}, we explained the cdf of the CND we made. This explanation helps us understand the distribution's features, but when it comes to sampling, the quantile function is key. In Proposition \ref{prop1} \parencite{awan2023canonical} provides a recursive formula for the CND's quantile function, as described in Definition \ref{def2}\parencite{awan2023canonical}, and show that we can finish it in just a few steps.

    \begin{prop}\label{prop1}   % Prop F.6
      Let $f$ be a symmetric nontrivial tradeoff function and let $F_f$ be as in Definition \ref{def1}. Then the quantile function $F_f^{-1}:(0,1)\rightarrow \RR$ for $F_f$ can be expressed as
      \[F_f^{-1}(u) = \begin{cases}
      F_f^{-1}(1-f(u))-1&u<c\\
      \frac{u-1/2}{1-2c}&c\leq u\leq 1-c\\
      F_f^{-1}(f(1-u))+1&u>1-c,
      \end{cases}\]
      where $c$ is the unique fixed point of $f$. {Furthermore, for any $u\in (0,1)$, the expression $Q_f(u)$ takes a finite number of recursive steps to evaluate. Thus,} if $U\sim U(0,1)$, then $F_f^{-1}(U) \sim F_f$. 
    \end{prop}
    
    % Cor 3.10
    According to Corollary 3.10 in \cite{awan2023canonical}, the distribution $\mathrm{Tulap}(0,b,q)$, where $b=\exp(-\ep)$ and $q = \frac{2\de b}{1-b+2\de b}$ is a CND for $f_{\ep,\de}$-DP, which agrees with the construction of Definition \ref{def1} \parencite{awan2023canonical}. 
    
    From the definition, it is easy to verify that the cdf of a Tulap random variable agrees with $F_f$ on $[-1/2,1/2]$. Tulap cdf also satisfies the recurrence relation of Definition \ref{def1} \parencite{awan2023canonical}. 
    Note that the cdf of $\mathrm{Tulap}(0,b,0)$ is 
    \[F_{N_0}(x) = \begin{cases}
    \frac{b^{-[x]}}{1+b}(b+\{x-[x]+1/2\}(1-b))& x\leq 0\\
    1- \frac{b^{[x]}}{1+b}(b+\{[x]-x+1/2\}(1-b))&x>0,
    \end{cases}\]
    where $[x]$ is the nearest integer function. The cdf of $\mathrm{Tulap}(0,b,q)$ is
    \[F_N(x) = \begin{cases}
    0&F_{N_0}(x)<q/2\\
    \frac{F_{N_0}(x)-q/2}{1-q}& q/2\leq F_{N_0}(x)\leq 1-q/2\\
    1&F_{N_0}(x)>1-q/2.
    \end{cases}\]
    By inspection, the fixed point of $f_{\epsilon,\delta}$ is $c=\frac{1-\delta}{1+e^\epsilon}$. It is easy to verify that $F_N(x) = c(1/2-x) + (1-c)(x+1/2)$ for $x\in (-1/2,1/2)$. 
    %By \cite[Lemma 2.8]{awan2020differentially},
    We then have that $F_N$ satisfies the recurrence relation in Definition \ref{def1} \parencite{awan2023canonical}. We conclude that $F_N = F_f$ and that $F_N$ is a CND for N. Therefore, The distribution $\mathrm{Tulap}(0,b,q)$ is privacy guaranteed.
\end{enumerate}
\end{proof}

\printbibliography

\end{document}
