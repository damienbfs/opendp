\documentclass{article}
% common styling and macros shared by all proof files

\usepackage[top=1in, right=1in, left=1in, bottom=1.5in]{geometry}

\usepackage{amsmath,amsthm,amsfonts,amssymb,amscd}
\usepackage{listings}
\usepackage{hyperref}
\usepackage{xcolor}
\usepackage{xr}

\usepackage{enumerate} 
\usepackage{physics}
\usepackage{fancyhdr}
\usepackage{hyperref}
\usepackage{graphicx}
\usepackage{tcolorbox}
\usepackage{catchfile}
\usepackage{pdftexcmds}
\usepackage[T1]{fontenc}

% hyperref
\hypersetup{
  colorlinks=true,
  linkcolor=blue,
  linkbordercolor={0 0 1}
}

% \contrib macro to indicate inclusion in "contrib".
\usepackage{tcolorbox}
\newtcolorbox{contrib_box}{colback=red!5!white,colframe=red!75!black}
\newcommand{\contrib}{{\begin{contrib_box}This proof resides in \textbf{``contrib''} because it has not completed the vetting process.\end{contrib_box}}} 

% asOfCommit macro to version a code dependency. Arguments:
%    #1: relative path to file you are dependent on
%    #2: commit hash it was last edited. If outdated, this should be the second hash in the footnoote. Otherwise,
%            git log -n 1 --pretty=format:%h -- path/to/file.rs
\makeatletter
\ifnum\pdf@shellescape=1
   % "private" command that builds a link to a blob
  \newcommand{\linkOpendpBlob}[3]{%
    \href{https://github.com/opendp/opendp/blob/#1/#2#3}{\path{#3} at commit #1}}

  % latex macro expansion has a separate phase for \input evaluation
  %     immediately evaluate a command to write a temp file to ./out containing the current directory
  \immediate\write18{[ ! -f out/cwd.txt ] && (mkdir -p out && git rev-parse --show-prefix | sed "s|_|\@backslashchar\@backslashchar\@backslashchar_|g" > out/cwd.txt)}
  %     ...and then retrieve the current working directory by loading the temp file
  \CatchFileDef\GitWorkingDir{out/cwd.txt}{\endlinechar=-1}

  % command for building the (up to date) or (outdated) status
  \newcommand{\fileStatus}[2]{%
  \setbox0=\hbox{\input|"git --no-pager log -n1 --pretty='\@percentchar H' #1 | grep -E '^#2.*'"\unskip}\ifdim\wd0=0pt
        (outdated\footnote{See new changes with \texttt{git diff #2..\input|"git --no-pager log -n1 --pretty='\@percentchar h' #1" \GitWorkingDir\path{#1}}})\else
        (up to date)\fi
  }

  \newcommand{\asOfCommit}[2]{%
      % permalink the target
      \linkOpendpBlob{#2}{\GitWorkingDir}{#1}
      % conditionally add (outdated) or (up to date) depending on matching commit hash
      \fileStatus{#1}{#2}%
  }
\else
  % simplified command if shell-escape not enabled
  \newcommand{\asOfCommit}[2]{#1 at commit #2 (unknown status\footnote{Shell-escape is not enabled. Enable \texttt{--shell-escape} to check if this proof is up-to-date with the code.})}
\fi
\makeatother

% \vettingPR macro to link a PR. Arguments:
%    #1: PR number
\newcommand{\vettingPR}[1]{\href{https://github.com/opendp/opendp/pull/#1}{Pull Request \##1}}

% \OpenDPVersion macro to get library version. Mainly used for \rustdoc
\makeatletter
\ifnum\pdf@shellescape=1
  % latex macro expansion has a separate phase for \input evaluation
  %     immediately evaluate a command to write a temp file to ./out containing the current version
  \immediate\write18{[ ! -f out/version.txt ] && (mkdir -p out && head -n 1 `git rev-parse --show-toplevel`/VERSION | sed "s|0.0.0+development|latest|g" > out/version.txt)}
  %     ...and then retrieve the version by loading the temp file
  \CatchFileDef\OpenDPVersion{out/version.txt}{\endlinechar=-1}
\else
  % if shell commands are not enabled, just claim latest
  \newcommand{\OpenDPVersion}{latest}
\fi
\makeatother

% for links to rustdoc items in OpenDP. Arguments:
%    #1: path to item, and designation as trait, struct, fn, etc.
%    #2: item name
\newcommand{\rustdoc}[2]{\href{https://docs.rs/opendp/\OpenDPVersion/opendp/#1.#2.html}{\texttt{#2}}}

% for links to external dependencies. Arguments:
%    #1: crate name
%    #2: path to item, and designation as trait, struct, fn, etc.
%    #3: item name
\newcommand{\docsrs}[3]{\href{https://docs.rs/#1/latest/#1/#2.#3.html}{\texttt{#3}}}

% minted (pseudocode)
\definecolor{codegreen}{rgb}{0,0.6,0}
\definecolor{codegray}{rgb}{0.5,0.5,0.5}
\definecolor{codepurple}{rgb}{0.58,0,0.82}
\definecolor{backcolour}{rgb}{0.95,0.95,0.92}

\lstdefinestyle{mystyle}{
    backgroundcolor=\color{backcolour},   
    commentstyle=\color{codegreen},
    keywordstyle=\color{magenta},
    numberstyle=\tiny\color{codegray},
    stringstyle=\color{codepurple},
    basicstyle=\ttfamily\footnotesize,
    breakatwhitespace=false,         
    breaklines=true,                 
    captionpos=b,                    
    keepspaces=true,                 
    numbers=left,                    
    numbersep=5pt,                  
    showspaces=false,                
    showstringspaces=false,
    showtabs=false,                  
    tabsize=2
}

\lstset{style=mystyle}

% common commands
\theoremstyle{definition}
\newtheorem{theorem}{Theorem}[section]
\newtheorem{lemma}[theorem]{Lemma}
\newtheorem{definition}[theorem]{Definition}
\newtheorem{warning}{Warning}
\newtheorem{corollary}{Corollary}
\newtheorem{proposition}{Proposition}
\newtheorem{remark}{Remark}
\newtheorem{observation}{Observation}
\newtheorem{note}{Note}

\newcommand{\vicki}[1]{{ {\color{olive}{(vicki)~#1}}}}
\newcommand{\hanwen}[1]{{ {\color{purple}{(hanwen)~#1}}}}
\newcommand{\zach}[1]{{ {\color{red}{(zach)~#1}}}}

\newcommand{\MultiSet}{\mathrm{MultiSet}}
\newcommand{\len}{\mathrm{len}}
\newcommand{\din}{\texttt{d\_in}}
\newcommand{\dout}{\texttt{d\_out}}
\newcommand{\T}{\texttt{T} }
\newcommand{\F}{\texttt{F} }
\newcommand{\Map}{\texttt{Map}}
\newcommand{\X}{\mathcal{X}}
\newcommand{\Y}{\mathcal{Y}}
\newcommand{\True}{\texttt{True}}
\newcommand{\False}{\texttt{False}}
\newcommand{\clamp}{\texttt{clamp}}
\newcommand{\function}{\texttt{function}}
\newcommand{\float}{\texttt{float }}
\newcommand{\questionc}[1]{\textcolor{red}{\textbf{Question:} #1}}


\newcommand{\transformationTheorem}[2]{%
  \begin{theorem}
    For every setting of the input parameters #1 to #2 such that the given preconditions
    hold, #2 raises an exception (at compile time or run time) or returns a valid transformation. A valid transformation has the following properties:
    \begin{enumerate}
        \item \textup{(Appropriate output domain).} 
        For every element $v$ in \texttt{input\_domain}, $\function(v)$ is in \texttt{output\_domain} or raises a data-independent runtime exception.
        
        \item \textup{(Domain-metric compatibility).} 
        The domain \texttt{input\_domain} matches one of the possible domains listed in the definition of \texttt{input\_metric}, 
        and likewise \texttt{output\_domain} matches one of the possible domains listed in the definition of \texttt{output\_metric}.
        
        \item \textup{(Stability guarantee).} 
        For every pair of elements $u, v$ in \texttt{input\_domain} and for every pair $(\din, \dout)$, 
        where \din\ has the associated type for \texttt{input\_metric} and \dout\ has the associated type for \\ \texttt{output\_metric}, 
        if $u, v$ are \din-close under \texttt{input\_metric} and $\texttt{stability\_map}(\din) \leq \dout$, 
        then $\function(u), \function(v)$ are $\dout$-close under \texttt{output\_metric}.
    \end{enumerate}
  \end{theorem}
}



\title{\texttt{fn make\_rappor}}
\author{Abigail Gentle}
\begin{document}

\maketitle



\contrib

Proves soundness of \rustdoc{measurements/fn}{make\_rappor} in \asOfCommit{mod.rs}{cfd1bec5}.

\section{Introduction}
RAPPOR is a protocol for local-differentially private frequency estimation. In the local model each user is guaranteed $\varepsilon$-differential privacy for their response. This is achieved by computing the \texttt{xor} of each input vector with a noise vector. Because the noise is added mechanically we can efficiently account for the bias introduced with the function \rustdoc{measurements/fn}{debias\_basic\_rappor}, which sums and normalises a vector of private outputs.

In the simplest case, each category is represented by an index $i\in[k]$, where $[k]=\{0,1,\ldots,k-1\}$, and each row with category $i$ is transformed into a one-hot vector which has 0's everywhere and a 1 at index $i$. Therefore the number of set bits in the input is 1, and the input domain should be a \rustdoc{domains/struct}{BitVectorDomain} with $\texttt{max\_weight}=1$

In order to estimate the frequencies of strings drawn from a potentially unbounded set, inputs can also be hashed onto a Bloom filter using $h<k$ hash functions. As such the number of set bits is at most $h$ and the input domain should be a \rustdoc{domains/struct}{BitVectorDomain} with $\texttt{max\_weight}=h$. As we demonstrate in Theorem~\ref{thm:privacy-parameter}, the privacy of the protocol will degrade linearly with the number of hash functions used (although the hash functions will make the life of an adversary trickier, as each bit set in the Bloom filter could map to multiple possible strings). 
\section{Hoare Triple}
\subsection{Preconditions}
\begin{itemize}
	\item Variable \texttt{input\_domain} must be of type \rustdoc{domains/struct}{BitVectorDomain}, with \texttt{max\_weight}.
	\item Variable \texttt{input\_metric} must be of type \rustdoc{metrics/struct}{DiscreteDistance}.
	\item Variable \texttt{f} must be of type \texttt{f64}.
    \item Variable \texttt{constant\_time} must be of type \texttt{bool}.
\end{itemize}

\subsection*{Pseudocode}
\begin{lstlisting}[language=Python, escapechar=|]
def make_rappor(f: float64, constant_time: bool):
    input_domain: BitVectorDomain
    input_metric: DiscreteDistance
    output_domain = BitVectorDomain
    output_measure = MaxDivergence(f64)
    
    if (f <= 0.0 or f > 1): |\label{line:range}|
        raise Exception("Probability must be in (0.0, 1]")
    
    m = input_domain.max_weight
    if m == None: 
        raise Exception("RAPPOR requires a maximum number of set bits")
    eps = (2*m)*log((2-f)/f)
    def privacy_map(d_in: IntDistance): |\label{line:map}|
        return eps
    def function(arg: BitVector) -> BitVector: |\label{line:fn}|
    	k = len(arg)
    	noise_vector = [bool.sample_bernoulli(f/2, constant_time) for _ in range(k)]
    	return xor(arg, noise_vector)
    
    return Measurement(input_domain, function, input_metric, output_measure, privacy_map)
\end{lstlisting}

\subsection*{Postcondition}
\validMeasurement{\texttt{(f, m, constant\_time)}}{\\ \texttt{make\_rappor}}


\section{Proof}
\begin{enumerate}
	\item Privacy guarantee
\begin{tcolorbox}
\begin{note}[Proof relies on correctness of Bernoulli sampler]
The following proof makes use of the following lemma that asserts the correctness of the Bernoulli sampler function.
    \begin{lemma}
    If system entropy is not sufficient, \texttt{sample\_bernoulli} raises an error. 
    Otherwise, \texttt{sample\_bernoulli(f/2, constant\_time)}, the Bernoulli sampler function used in \texttt{make\_randomized\_response\_bool}, 
    returns \texttt{true} with probability (\texttt{prob}) and returns  \texttt{false} with probability (1 - \texttt{f/2}).
    \end{lemma}
\end{note}
\end{tcolorbox}
\begin{theorem}~\cite{rappor}
\label{thm:privacy-parameter}
	\rustdoc{measurements/fn}{make\_rappor} satisfies $\varepsilon$-DP where 
	\begin{equation*}
		\varepsilon = 2m\log\left(\frac{2-f}{f}\right)
	\end{equation*}
\end{theorem}
\begin{lemma}
	\begin{align}
		P[y_i = 1~|~x_i=1] &= 1 - \frac{1}{2}f\\
		P[y_i = 1~|~x_i=0] &=\frac{1}{2}f
	\end{align}
\end{lemma}
\begin{proof}
	Let $Y=y_1,\ldots,y_k$ be a randomised report generated by \rustdoc{measurements/fn}{make\_rappor}. Then the probability of observing any given report $Y$ is $P[Y=y | X=x]$. $x=x_1,\ldots,x_k$ is a single Boolean vector with at most $m$ ones. 
	Without loss of generality assume that $x^*=\{x_1=1,\ldots,x_m=1,x_{m+1}=0,\ldots,x_k=0\}$, then we have
	\begin{align*}
		P[Y=y~|~X=x^*] &=%
			\prod\limits_{i=1}^m \left(\frac{1}{2}f\right)^{1-y_i}\left(1-\frac{1}{2}f\right)^{y_i}%
			\times \prod\limits_{i=m+1}^k\left(\frac{1}{2}f\right)^{y_i} \left(1-\frac{1}{2}f\right)^{1-y_i}
	\end{align*}
	Then let $D$ be the ratio of two such conditional probabilities with distinct values $x_1$ and $x_2$, and let $S$ be the range of \rustdoc{measurements/fn}{make\_rappor}.
	\begin{align}
		D &= \frac{P[Y\in S~|~X=x_1]}{P[Y\in S~|~X=x_2]}\\
			&= \frac{\sum_{y\in S}P[Y=y~|~X=x_1]}{\sum_{y\in S}P[Y=y~|~X=x_2]}\nonumber\\
			&\leq \max_{y\in S}\frac{P[Y=y~|~X=x_1]}{P[Y=y~|~X=x_2]}\nonumber\\
			&=\max_{y\in S}\frac{%
				\prod\limits_{i=1}^m \left(\frac{1}{2}f\right)^{1-y_i}\left(1-\frac{1}{2}f\right)^{y_i}%
				\times \prod\limits_{i=m+1}^k\left(\frac{1}{2}f\right)^{y_i} \left(1-\frac{1}{2}f\right)^{1-y_i}
			}{%
				\prod\limits_{i=1}^m \left(\frac{1}{2}f\right)^{y_i}\left(1-\frac{1}{2}f\right)^{1-y_i}%
				\times \prod\limits_{i=m+1}^{2m}\left(\frac{1}{2}f\right)^{1-y_i} \left(1-\frac{1}{2}f\right)^{y_i}\times%
				\prod\limits_{i=2m+1}^{k}\left(\frac{1}{2}f\right)^{y_i}\left(1-\frac{1}{2}f\right)^{1-y_i}
			}\nonumber\\
			&=\max_{y\in S}\frac{%
				\prod\limits_{i=1}^m \left(\frac{1}{2}f\right)^{1-y_i}\left(1-\frac{1}{2}f\right)^{y_i}%
				\times \prod\limits_{i=m+1}^{2m}\left(\frac{1}{2}f\right)^{y_i} \left(1-\frac{1}{2}f\right)^{1-y_i}%
			}{%
				\prod\limits_{i=1}^m \left(\frac{1}{2}f\right)^{y_i}\left(1-\frac{1}{2}f\right)^{1-y_i}%
				\times \prod\limits_{i=m+1}^{2m}\left(\frac{1}{2}f\right)^{1-y_i} \left(1-\frac{1}{2}f\right)^{y_i}%
			}\label{eq:privacy:cancel-k}\\
			&=\max_{y\in S}\left[%
				\prod\limits_{i=1}^m \left(\frac{1}{2}f\right)^{2(1-y_i)}\left(1-\frac{1}{2}f\right)^{2y_i}%
				\times \prod\limits_{i=m+1}^{2m}\left(\frac{1}{2}f\right)^{2y_i} \left(1-\frac{1}{2}f\right)^{2(1-y_i)}\right]\label{eq:maximise-product}%
	\end{align}
	Notice that by equation~\ref{eq:privacy:cancel-k} the privacy is not dependent on $k$ and equation~\ref{eq:maximise-product} is maximised when $y_1=1,\ldots,y_m=1$, and $y_{m+1},\ldots,y_{2m}=0$, giving
	\begin{align*}
		D &\leq \left(1-\frac{1}{2}f\right)^{2m}\times\left(\frac{1}{2}f\right)^{-2m}\\
			&= \left(\frac{2-f}{f}\right)^{2m}
	\end{align*}
	Therefore,
	\begin{equation}
		\varepsilon \leq 2m\log\left(\frac{2-f}{f}\right)
	\end{equation}
\end{proof}
\item Utility
\begin{theorem}
\label{thm:unbiased-estimator}
	\rustdoc{measurements/fn}{debias\_basic\_rappor} is an unbiased frequency estimator.
\end{theorem}
\begin{proof} We denote the input domain as $X$, which is the set of all vectors with hamming weight less than or equal to $m$. The set of all users is $X^n$, where each user $x_1,\ldots,x_j,\ldots,x_n$ holds a single input. Each input $x_j$ is transformed into $y_j$ using \rustdoc{measurements/fn}{make\_rappor}. Let $Y$ be the sum of $n$ received outputs, where $Y_i=\sum_{j=1}^n y_i$ is the number of received bits at index $i\in [k]$. Let $N_i=\sum x_{j,i}$ be the true (non-private) count vector of users with bit $i$ set. Our goal is to estimate the frequencies $q_i=N_i/n$ with minimal error. $Y_i$ is a sum of two binomials
\begin{align}
\mathbb{E}[Y_i] &= \mathbb{E}\left[\text{Bin}\left(N_i,1-\frac{f}{2}\right)+\text{Bin}\left(n-N_i,\frac{f}{2}\right)\right]\label{eq:YsumBin}\\
    &=N_i\left(1-\frac{1}{2}f\right) + (n-N_i)\frac{f}{2}\label{eq:expec-Y}
\end{align}
Therefore by rearranging~\ref{eq:expec-Y} we obtain an unbiased estimator for $\hat{N_i}$
\begin{equation}
    \label{eq:estimator}
    \hat{N_i} = \frac{Y_i-n\frac{f}{2}}{1-f}
\end{equation}

Our goal however, is to estimate the frequency of each element $\hat{q}_i$ so we need to normalise by dividing by $n$ which gives us
\[
\hat{q_i}=\frac{\mathbb{E}[\hat{N}_i]}{n} = \frac{\frac{Y_i}{n}-\frac{f}{2}}{1-f}
\]
\end{proof}
\begin{theorem}
	\rustdoc{measurements/fn}{debias\_basic\_rappor} is a frequency estimator with mean squared error
	\begin{equation*}
		\mathbb{E}[\ell_2^2(q-\hat{q})] = \frac{k\left(\frac{f}{2}-f^2\right)}{4n(1-f)^2}.
	\end{equation*}
\end{theorem}
\begin{proof}\hfill

Let $q = q(X^n)=\sum\limits_{j=1}^n x_j$ be the non private frequency vector of inputs, our goal is to find the mean squared error of our estimate from this vector. Using the fact from equation~\ref{eq:YsumBin} that $Y_i$ is a sum of two Binomials with equal variance (by the symmetry of their probabilities), we get
\begin{equation}
\label{eq:varianceY}
    \text{Var}(Y_i)=n\frac{f}{2}\left(1-\frac{f}{2}\right)
\end{equation}
	\begin{align*}
		\mathbb{E}[|\hat{q}-q|_2^2] &= \mathbb{E}\left[\sum\limits_{i=1}^k(\hat{q_i}-q_i)^2\right] = \sum\limits_{i=1}^k\mathbb{E}[(\hat{q_i}-q_i)^2] &&\\
			&= \sum\limits_{i=1}^k\mathbb{E}[(\hat{q_i}-\mathbb{E}[\hat{q_i}])^2] && \text{by Theorem~\ref{thm:unbiased-estimator}}\\
			&= \sum\limits_{i=1}^k\text{Var}(\hat{q_i}) = \sum\limits_{i=1}^k\text{Var}\left(\frac{\frac{\hat{Y_i}}{n}-\frac{f}{2}}{1-f}\right)&&\text{by Equation~\ref{eq:estimator}}\\
			&= \sum\limits_{i=1}^k\text{Var}\left(\frac{Y_i}{n(1-f)}\right) &&\\
			&= \sum\limits_{i=1}^k\frac{\text{Var}(Y_i)}{n^2(1-f)^2}&&\\
			&= k\left(\frac{n\frac{f}{2}\left(1-\frac{f}{2}\right)}{n^2(1-f)^2}\right) &&\text{ by Equation~\ref{eq:varianceY}}\\
			& = \frac{k\left(f-\frac{f^2}{2}\right)}{2n(1-f)^2}
	\end{align*}
	
\end{proof}
\end{enumerate}

\bibliographystyle{plain}
\bibliography{references.bib}
\end{document}